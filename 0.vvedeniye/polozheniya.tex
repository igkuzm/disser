%% File              : polozheniya.tex
%% Author            : Igor V. Sementsov <ig.kuzm@gmail.com>
%% Date              : 05.08.2023
%% Last Modified Date: 05.08.2023
%% Last Modified By  : Igor V. Sementsov <ig.kuzm@gmail.com>

Основные положения, выносимые на защиту


     1.    Применение    технологий    3D-печати   в   хирургии   позвоночника
позволяет улучшить результаты хирургического лечения путем снижения числа
осложнений и ревизионных вмешательств, повышения точности установки
винтовых фиксирующих систем, сокращения времени операции и улучшенного
периоперационного    планирования      и   способствует   сокращению   кривой
обучаемости при освоении новых видов операций.
     2.    Применение индивидуальных навигационных направителей позволяет
выполнять установку транспедикулярных винтов в шейном отделе позвоночника
с высокими показателями точности и безопасности. Оптимальным дизайном
является билатеральный одноуровневый навигационный направитель с фиксацией
на верхушке остистого отростка по типу «ключ-к-замку» и опорной зоной,
частично покрывающей дорзальные структуры позвонка.
     3.    Метод индивидуальных навигационных направителей обеспечивает
более точную и быструю установку транспедикулярных винтов по сравнению с
интраоперационной КТ-навигацией в экспериментальном исследовании при
значительно меньших финансовых затратах.
     4.    Использование индивидуальных навигационных направителей при
винтовой фиксации С2 позвонка сопровождается лучшими показателями
безопасности имплантации и меньшим числом нейроваскулярных осложнений по
сравнению с методом «free hand».
     5.    Применение индивидуальных навигационных направителей в грудном
отделе позвоночника сопровождается лучшими показателями безопасности и
точности установки транспедикулярных винтов и снижает число ревизионных
вмешательств по сравнению с методом «free hand», при этом опора на остистый
отросток не оказывает влияния на показатели имплантации, рекомендуемым
дизайном является билатеральная одноуровневая матрица с частичной опорой на
дорзальные структуры позвонка.
     6.      Применение       индивидуальных        навигационных       направителей     в
пояснично-крестцовом отделе позвоночника при установке транспедикулярных
винтов     по     субкортикальной      траектории    сопровождается           аналогичными
показателями безопасности по сравнению с использованием интраоперационной
флуороскопии, при этом сокращает время имплантации и лучевую нагрузку.
     7.      Применение       индивидуальных       моделей   позвоночника         улучшает
показатели имплантации, сокращает время операции, операционную кровопотерю
и частоту повторных операций у начинающих хирургов, снижает лучевую
нагрузку     на    пациента     и   персонал   по    сравнению      с    использованием
интраоперационного флуороскопического контроля.


