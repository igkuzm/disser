%% File              : novizna.tex
%% Author            : Igor V. Sementsov <ig.kuzm@gmail.com>
%% Date              : 05.08.2023
%% Last Modified Date: 05.08.2023
%% Last Modified By  : Igor V. Sementsov <ig.kuzm@gmail.com>

Научная новизна

В ходе исследования впервые проведено сравнение безопасности и
точности   имплантации    винтовых    фиксирующих        систем    позвоночника    с
применением индивидуальных навигационных направителей различного дизайна
в шейном и грудном отделах позвоночника. Разработан оригинальный дизайн
индивидуальных навигационных направителей, обеспечивающий наилучшие
показатели имплантации в шейном и грудном отделах позвоночника (патент РФ
№ 198660, 2020 г.; патент РФ №200909, 2020 г.).
      Впервые проведено сравнение параметров имплантации с применением
интраоперационной     КТ-навигации        и       индивидуальных      навигационных
направителей по показателям безопасности и времени фиксации, лучевой
нагрузке и финансовых затратах. Проведен комплексный анализ девиации
фактической и планируемой траекторий имплантации при использовании
индивидуальных навигационных направителей во всех отделах позвоночника.
Выполнено сравнение безопасности и точности установки транспедикулярных
винтов в пояснично-крестцовом отделе позвоночника по субкортикальной
траектории с использованием индивидуальных навигационных направителей и
интраоперационной    флуороскопии.    Изучена        эффективность     использования
индивидуальных     моделей   позвоночника         при   типовых      декомпрессивно-
стабилизирующих операциях в пояснично-крестцовом отделе.
      Впервые изучено влияние использования индивидуальных моделей
позвоночника на качество и временные параметры выполненных операций в
зависимости от опыта хирурга.
