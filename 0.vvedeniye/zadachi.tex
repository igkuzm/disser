%% File              : zadachi.tex
%% Author            : Igor V. Sementsov <ig.kuzm@gmail.com>
%% Date              : 05.08.2023
%% Last Modified Date: 05.08.2023
%% Last Modified By  : Igor V. Sementsov <ig.kuzm@gmail.com>

Задачи исследования

\item Провести доклиническую оценку безопасности и точности установки
			транспедикулярных винтов и определить оптимальные параметры дизайна
			индивидуальных навигационных направителей в шейном и грудном отделах
			позвоночника по результатам кадавер-эксперимента.

\item Провести   сравнительную    оценку   показателей
			имплантации   с
			использованием индивидуальных навигационных направителей и
			интраоперационной КТ-навигации в эксперименте на биомакетах крупного
			лабораторного
			животного.

\item Определить безопасность и точность имплантации
			транспедикулярных винтов в С2 позвонок с использованием индивидуальных
			навигационных
			направителей в сравнении с методом «free hand» в клиническом
			исследовании.

\item Определить безопасность и точность имплантации
			транспедикулярных винтов в шейном отделе позвоночника на
			субаксиальных уровнях с
			использованием индивидуальных навигационных направителей
			в клиническом
			исследовании.

\item Определить безопасность и точность
			имплантации транспедикулярных винтов с использованием индивидуальных
			навигационных направителей различного дизайна в грудном отделе
			позвоночника по сравнению с
			методом «free hand» в клиническом исследовании.

\item Определить безопасность и точность
			имплантации транспедикулярных винтов с использованием
			индивидуальных навигационных направителей
			в пояснично-крестцовом отделе позвоночника
			по сравнению с применением
			интраоперационной флуороскопии в клиническом
			исследовании.

\item Определить влияние применения
			индивидуальных моделей позвоночника на параметры операции и
			результаты хирургического лечения в
			зависимости от индивидуального опыта
			оперирующего хирурга.

