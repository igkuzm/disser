%% File              : metodi.tex
%% Author            : Igor V. Sementsov <ig.kuzm@gmail.com>
%% Date              : 05.08.2023
%% Last Modified Date: 05.08.2023
%% Last Modified By  : Igor V. Sementsov <ig.kuzm@gmail.com>

Методология и методы исследования

Диссертационная работа основана на серии доклинических и клинических
исследований применения 3D-моделей и индивидуальных навигационных
направителей при операциях на позвоночнике. Проектирование объектов
осуществлялось на основе данных КТ, КТ-ангиографии и МРТ. Для обработки
исходных DICOM-файлов, проектирования и печати использовались программы
Инобитек DICOM Просмотрщик Профессиональная Редакция 1.9.0., MIMICS
Research 20.0, Horos version 3.1.1., Blender 2.78, Cura 15.04. Печать выполнялась
на трех различных принтерах технологией струйного наложения расплавленной
полимерной нити (FDM) из материалов PLA, PVA, HIPS, Flex.
     Оценка       безопасности   имплантации      проводилась      по    данным
послеоперационной КТ и регистрации периоперационных осложнений, точности
имплантации – путем оценки девиации планируемой и фактической траекторий в
программе Mimics 3D.
     В эксперименте на кадавер-препаратах шейного и грудного отделов
позвоночника произведена доклиническая оценка безопасности и точности
имплантации транспедикулярных винтов, определен дизайн индивидуальных
навигационных направителей, обеспечивающий лучшие показатели установки.
     В   эксперименте    на   биомакетах    грудного   и   поясничного   отделов
позвоночника барана выполнено сравнение имплантации транспедикулярных
винтов с использованием индивидуальных навигационных направителей и
интраоперационной КТ-навигации по показателям безопасности имплантации,
времени установки, лучевой нагрузке и финансовых затратах. Произведен расчет
времени на весь цикл производства направителей и моделей позвоночного столба.
     В    серии    клинических   исследований     определена    безопасность   и
эффективность применения технологий 3D–печати, преимущества и недостатки
метода во всех отделах позвоночника.
     Сравнительный анализ безопасности и точности имплантации винтов в С2
позвонок с применением индивидуальных 3D-навигационных направителей и по
методике «free hand» выполнена в ходе нерандомизированного контролируемого
исследования. В опытной группе выполнялась установка винтов с применением
направителей (21 пациент, 42 винта). В контрольной группе (23 пациента, 44
винта) был проведен ретроспективный анализ данных пациентов, которым
проводилась имплантация винтов в С2 позвонок по методике «free hand» в 2010-
2016 гг.
      Для оценки точности и безопасности выполнения транспедикулярной
фиксации в шейном отделе позвоночника с использованием индивидуальных
навигационных направителей на субаксиальных уровнях было выполнено
неконтролируемое исследование с имплантацией 127 винтов 28 пациентам в 2017-
2020 гг.
      Оценка безопасности и точности установки винтов в грудном отделе
позвоночника    и   определение   приоритетного    дизайна     индивидуальных
навигационных направителей выполнено в сравнительном исследовании с
имплантацией 208 транспедикулярных винтов 47 пациентам в период 2018-2020
гг. с формированием трех групп: группа 1 — имплантация винтов по методике
«free hand», группа 2 — имплантация винтов с помощью двусторонних
индивидуальных навигационных направителей, группа 3 — имплантация винтов с
помощью индивидуальных навигационых направителей с трехточечной опорой.
      Анализ эффективности и безопасности применения индивидуальныых
навигационных направителей в пояснично-крестцовом отделе позвоночника
выполнен в одноцентровом рандомизированном сравнительном исследовании с
имплантацией 130 транспедикулярных винтов 29 пациентам при декомпрессивно-
стабилизирующих операциях в пояснично-крестцовом отделе позвоночника по
методике   MIDLIF    в   сравнении    с   использованием     интраоперационной
флуороскопии.
      Для оценки эффективности использования индивидуальных 3D-моделей
позвоночника у хирургов с различным персональным опытом выполнено
рандомизированное контролируемое исследование с анализом интраоперацион-
ных параметров и осложнений типовых декомпрессивно-стабилизирующих
операций по методике TLIF, проведенных 71 пациенту в 2016-2020 гг. с
формированием четырех групп сравнения.

