%% File              : znachimost.tex
%% Author            : Igor V. Sementsov <ig.kuzm@gmail.com>
%% Date              : 05.08.2023
%% Last Modified Date: 05.08.2023
%% Last Modified By  : Igor V. Sementsov <ig.kuzm@gmail.com>

Теоретическая и практическая значимость работы

В ходе исследования изучена эффективность и безопасность имплантации
винтовых   стабилизирующих      систем        с   использованием     индивидуальных
навигационных направителей во всех отделах позвоночника при различных
патологических процессах.
      Согласно опубликованным данным, представленный опыт имплантации
винтовых систем с использованием индивидуальных навигационных направи-
телей представляет наибольшую серию в РФ и одну из наибольших в мире.
      Доказано, что использование технологий 3D-печати позволяет улучшить
результаты хирургического лечения пациентов с заболеваниями и травмами
позвоночника и повысить точность имплантации металлоконструкций.
      Произведен комплексный анализ девиации траекторий имплантации при
использовании индивидуальных навигационных направителей в зависимости от
их дизайна, уровня фиксации и других факторов.
      Выполнено сравнение метода спинальной навигации с использованием
индивидуальных навигационных направителей с другими актуальными методами
установки винтовых систем, в том числе, интраоперационной КТ-навигацией.
      Произведен расчет временных показателей и финансовых затрат на весь
цикл изготовления индивидуальных моделей и направителей от момента
получения КТ-данных до их применения в операционной.
      Определены     преимущества        использования   индивидуальных    3D-
биомоделей   при   выполнении    типовых      декомпрессивно-стабилизирующих
операций на поясничном отделе позвоночника у хирургов с различным
персональным опытом.
      Разработаны практические рекомендации по проектированию, печати и
применению индивидуальных 3D-моделей и навигационных направителей при
операциях на всех отделах позвоночника.



