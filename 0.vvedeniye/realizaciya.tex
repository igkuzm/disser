%% File              : realizaciya.tex
%% Author            : Igor V. Sementsov <ig.kuzm@gmail.com>
%% Date              : 05.08.2023
%% Last Modified Date: 05.08.2023
%% Last Modified By  : Igor V. Sementsov <ig.kuzm@gmail.com>


         Реализация результатов исследования

	Применение индивидуальных 3D-моделей позвоночника и индиви-
дуальных        навигационных       направителей     используется       для     повышения
эффективности периоперационного планирования, повышения точности и
безопасности имплантации винтовых фиксирующих систем при выполнении
оперативных вмешательств на позвоночнике в условиях нейрохирургических и
травматолого-ортопедических отделений ФГБУ «НМИЦ им. В.А.Алмазова»
Минздрава России, ФГБУ «НМИЦ ТО им. Р.Р. Вредена» Минздрава России,
клиники нейрохирургии ФГБОУ ВПО «ПСПбГМУ им. И.П.Павлова» Минздрава
России, ФГБУ «Федеральный центр нейрохирургии» Минздрава России, СПб
ГБУЗ "Городская многопрофильная больница №2", ФГБУ «НМИЦ травматологии
и ортопедии имени академика Г.А. Илизарова» Минздрава России, ФГБУ «ГНЦ
ФМБЦ им. А.И.Бурназяна» ФМБА России. Материалы диссертационного
исследования используются при обучении клинических ординаторов, а также при
чтении лекций и проведении семинаров в рамках подготовки по специальностям
«нейрохирургия» и «травматология и ортопедия» студентам медицинским ВУЗов
и практикующим врачам на кафедре нейрохирургии нейрохирургии ФГБОУ ВПО
«ПСПбГМУ им. И.П.Павлова» Минздрава России, кафедре оперативной хирургии
и топографической анатомии имени проф. В.И.Валькера ФГБУ ВО «Санкт-
Петербургский государственный педиатрический медицинский университет»
Минздрава России.


