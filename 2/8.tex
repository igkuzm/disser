%% File              : 8.tex
%% Author            : Igor V. Sementsov <ig.kuzm@gmail.com>
%% Date              : 05.08.2023
%% Last Modified Date: 05.08.2023
%% Last Modified By  : Igor V. Sementsov <ig.kuzm@gmail.com>

Материал и методы клинических исследований 


Сравнительный анализ безопасности и эффективности имплантации
винтов в С2 позвонок с применением индивидуальных навигационных
направителей и по методике «free hand»
.................................................................. 107
2.8.2. Анализ безопасности и эффективности имплантации транспедикулярных
винтов на шейных субаксиальных уровнях c использованием индивидуальных
навигационных направителей
................................................................................... 108
2.8.3. Сравнение безопасности и эффективности имплантации
транспедикулярных винтов в грудном отделе позвоночника с иcпользованием
индивидуальных навигационных направителей различного дизайна по
сравнению с методикой «free
hand».......................................................................... 111
2.8.4. Сравнение безопасности и эффективности имплантации
транспедикулярных винтов в поясничном отделе по субкортикальной
траектории с использованием индивидуальных навигационных направителей
и интраоперационной флуороскопии
....................................................................... 114
2.8.5. Эффективность использования индивидуальных 3D-моделей
позвоночника при декомпрессивно-стабилизирующих операциях в пояснично-
крестцовом отделе в зависимости от персонального опыта хирурга .
