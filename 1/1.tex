%% File              : 1.tex
%% Author            : Igor V. Sementsov <ig.kuzm@gmail.com>
%% Date              : 05.08.2023
%% Last Modified Date: 11.03.2024
%% Last Modified By  : Igor V. Sementsov <ig.kuzm@gmail.com>

Применение кастомизированных имплантов в хирургии 

В настоящее время, в хирургии широко променяются различные катомизированные
импланты.

Изготовленные на заказ трехфланцевые вертлужные кейджи, напечатанные на
3D-принтере, являются новой опцией в арсенале ревизионного хирурга
тазобедренного сустава. в этой обзорной статье описываются эта технология, ее
показания, хирургический метод, преимущества, недостатки, использование, текущая
опубликованная литература и будущие применения.\cite{wyatt2015}

Введение: сегодня при ревизионном тотальном эндопротезировании тазобедренного
сустава (РТА) используются различные варианты реконструкции потери вертлужной
кости. Целью исследования было сравнение результатов использования стандартных
вертлужных имплантатов (sais) и индивидуальных вертлужных имплантатов (cmais)
при арта в случаях с обширной потерей вертлужной кости. методы: проведен
сравнительный анализ результатов 106 операций РТА, выполненных за период с
января 2013 г. по декабрь 2019 г. В 61 случае (57,5\%) применялся КМАИС. в 45
случаях (42,5\%) саи были имплантированы. Результаты: частота асептического
расшатывания вертлужного компонента после РТА при неконтролируемой потере
костного фонда вертлужной впадины (тип III-IV по классификации Gross и
Saleh) при использовании cmai была меньше, чем при использовании sai (2,4\% и
10,0\%)  соответственно). наиболее значимые различия в показателях
асептического расшатывания отмечены после имплантации cmai и sai при
несплошности таза с неконтролируемым костным дефектом (0\% и 60,0\%
соответственно; p<0,001). Заключение: идеальными показаниями к использованию
ЦМАИ являются неконтролируемые дефекты и несплошность таза с
неконтролируемой потерей костного материала (типы III-V по общей
классификации и классификации Салеха). обработка этих дефектов саем приводит
к более высокой частоте асептического расшатывания, требующего повторных
ревизий. требуется дальнейшее наблюдение для оценки эффективности
использования cmai и sai в долгосрочном периоде
наблюдения.\cite{tikhilov2022}

Применение трехмерной (3D) печати или аддитивного производства в области
хирургии позвоночника продолжает расти в количестве и масштабах, особенно в
последние годы, поскольку усовершенствованные технологии производства и
использование стерилизуемых материалов позволили создать 3D-печатные имплантаты.
Хотя 3D-печать в хирургии позвоночника изначально ограничивалась использованием
в качестве наглядного пособия при предоперационном планировании сложных
патологий, в последнее время она стала использоваться для создания
интраоперационных направляющих и шаблонов винтов для конкретного пациента и все
чаще используется в хирургическом обучении и обучении. Поскольку индивидуальное
лечение и персонализированная медицина набирают популярность в медицине,
3D-печать предоставляет аналогичную возможность для хирургических полей,
особенно при создании настраиваемых имплантатов. 3D-печать — относительно новая
область в хирургии позвоночника, поэтому в ней отсутствуют долгосрочные данные о
клинических результатах и экономической эффективности; однако
очевидные преимущества и, казалось бы, безграничные возможности применения этой
развивающейся технологии делают ее привлекательным вариантом для будущей
хирургии позвоночника.\cite{sheha2019}

Использование 3D-печати приобретает значительный успех во многих областях
медицины, включая хирургию. здесь была внедрена технология повышения уровня
анатомического понимания благодаря характеристикам, присущим 3D-печатным
моделям: это высокоточные и индивидуальные репродукции, полученные на основе
собственных радиологических изображений пациентов, и представляющие собой
твердые захватываемые объекты, позволяющие свободно манипулировать ими. часть
пользователя. возникающая в результате тактильная обратная связь значительно
помогает пониманию анатомических деталей, особенно пространственных отношений
между структурами. в этом отношении они оказались более эффективными, чем
традиционные 2D-изображения и виртуальные 3D-модели. на сегодняшний день все
большее число приложений успешно протестировано во многих хирургических
дисциплинах, расширяя диапазон возможных применений до предоперационного
планирования, консультирования пациентов, обучения студентов и ординаторов,
хирургической подготовки, интраоперационной навигации и других; в последние годы
3D-печать также применяется для создания хирургических инструментов и
воспроизведения анатомических частей, которые будут использоваться
соответственно в качестве шаблонов или направляющих для конкретных задач
хирургии и индивидуализированных имплантируемых материалов в реконструктивных
процедурах. Будущие ожидания касаются, с одной стороны, сокращения
производственных затрат и времени для дальнейшего повышения доступности
3D-печати, а с другой - разработки новых методов и материалов, подходящих для
3D-печати биологических структур, с помощью которых воссоздается архитектура и
функциональность реального человека. органы и ткани.\cite{pugliese2018}

Цель: данное исследование было направлено на оценку эффективности трехмерной
(3D) печати при обычных операциях при переломах проксимального отдела плечевой
кости (ППП). методы: в восьми базах данных был проведен комплексный поиск данных
о клинических характеристиках и результатах, включая время операции, время
заживления кости, объем кровопотери, количество интраоперационных флюороскопий,
скорость редукции анатомических проксимальных отделов плечевой кости, постоянные
баллы, рейтинг neer, потерю плечевой кости. высота головы и осложнения. эти
данные сравнивались между операциями с использованием 3D-печати и традиционными
операциями, чтобы оценить эффективность хирургии с использованием 3D-печати.
Результаты: хирургия с использованием 3D-печати превзошла традиционные процедуры
по времени операции, объему кровопотери, времени до сращения phfs, количеству
флюороскопий, скорости редукции анатомических проксимальных отделов плечевой
кости, постоянным баллам, рейтингу NEER и осложнениям. Вывод: хирургия с
использованием 3D-печати сокращает время операции, анатомическое заживление,
уменьшает боль и движение, нанося меньший вред пациентам.\cite{li2022}

Предыстория: трехмерная (3D) печать имеет множество применений и вызвала большой
интерес в мире медицины. Постоянно улучшающееся качество приложений 3D-печати
способствовало их более широкому использованию пациентами. В этой статье
обобщается литература по хирургическим применениям 3D-печати, используемым на
пациентах, с акцентом на сообщаемые клинические и экономические результаты.
методы: были проверены три основные базы данных литературы на предмет серий
случаев (более трех случаев, описанных в одном исследовании) и испытаний
хирургического применения 3D-печати на людях. Результаты: 227 хирургических
статей были проанализированы и обобщены с использованием таблицы доказательств.
в документах описывалось использование 3D-печати для хирургических шаблонов,
анатомических моделей и индивидуальных имплантатов. 3D-печать используется во
многих хирургических областях, таких как ортопедия, челюстно-лицевая хирургия,
черепная хирургия и хирургия позвоночника. В целом, к преимуществам деталей,
напечатанных на 3D-принтере, относятся сокращение времени хирургического
вмешательства, улучшение медицинских результатов и снижение радиационного
воздействия. затраты на печать и дополнительное сканирование обычно увеличивают
общую стоимость процедуры. Вывод: 3D-печать хорошо интегрирована в хирургическую
практику и исследования. Применение варьируется от анатомических моделей,
предназначенных в основном для хирургического планирования, до хирургических
шаблонов и имплантатов. Наше исследование показывает, что у 3D-печатных
приложений есть несколько преимуществ, но необходимы дальнейшие исследования,
чтобы определить, можно ли сбалансировать возросшие затраты на вмешательство с
наблюдаемыми преимуществами этой новой технологии. существует необходимость в
формальном анализе экономической эффективности.\cite{tack2016}

Предыстория: трехмерная (3D) печать — это революционная технология, которая
быстро распространяется во многих областях, включая здравоохранение. в этом
контексте он позволяет создавать понятные, индивидуальные анатомические модели,
созданные на основе медицинских изображений. способность держать и показывать
физический объект ускоряет и облегчает понимание анатомических деталей,
облегчает консультирование пациентов и способствует образованию и обучению
студентов и ординаторов. несколько медицинских специальностей в настоящее время
изучают потенциал этой технологии, включая общую хирургию. методы: в этом обзоре
мы даем обзор доступных технологий 3D-печати, а также систематический анализ
медицинской литературы, посвященной ее применению в абдоминальной хирургии.
Также сообщается о нашем опыте работы с первой клинической лабораторией
3D-печати в Италии. Результаты: за последнее десятилетие произошло десятикратное
увеличение количества публикаций в год. около 70\% этих статей посвящены моделям
почек и печени и созданы в первую очередь для предварительного планирования, а
также в образовательных и учебных целях. Наиболее распространенными технологиями
печати являются струйная печать и экструзия материала. семьдесят три процента
публикаций сообщали о менее чем десяти клинических случаях. Вывод: растущее
применение 3D-печати в абдоминальной хирургии отражает зарождение новой
технологии, хотя она все еще находится в зачаточном состоянии. Однако
потенциальная польза от этой технологии очевидна, и вскоре она может привести к
созданию новых больничных учреждений для улучшения хирургической подготовки,
исследований и ухода за пациентами.\cite{pietrabissa2020}

усовершенствования технологий и снижение затрат привели к широкому интересу к
трехмерной (3D) печати. Анатомические модели, напечатанные на 3D-принтере,
способствуют персонализированной медицине, хирургическому планированию и
образованию по медицинским специальностям, и эти модели быстро меняют ландшафт
клинической практики. Физический объект, который можно держать в руках, дает
значительные преимущества перед стандартными двумерными (2D) или даже
трехмерными виртуальными моделями, созданными на компьютере. Радиологи могут
сыграть значительную роль в качестве консультантов и преподавателей по всем
специальностям, предоставляя 3D-печатные модели, которые улучшают клиническую
помощь. В этой статье рассматриваются основы 3D-печати, в том числе способы
создания моделей на основе данных визуализации, клиническое применение 3D-печати
в области живота и таза, значение для образования и обучения, ограничения и
будущие направления.\cite{bastawrous2018}

**цели**: медицинская трехмерная (3D) печать, изготовление портативных моделей
на основе медицинских изображений, потенциально может стать неотъемлемой частью
отоларингологической хирургии головы и шеи (oto-hns) с широким влиянием на все
ее узкоспециализированные области. . мы рассматриваем основные принципы этой
технологии и предоставляем подробное описание клинических применений в этой
области. методы: с момента их создания до мая 2018 года в стандартных
библиографических базах данных (Medline, Embase, совокупный указатель литературы
по сестринскому делу и сопутствующей медицинской литературе, Web of Science и
Кокрановскому центральному реестру рандомизированных исследований) проводился
поиск по терминам: «3D-печать». трехмерная печать», «быстрое прототипирование»,
«аддитивное производство», «компьютерное проектирование», «биопечать» и
«биопроизводство» в различных сочетаниях с терминами: «птоларингология»,
«хирургия головы и шеи» и «отология». дополнительные статьи были
идентифицированы по ссылкам на найденные статьи. были включены только
исследования, описывающие клиническое применение 3D-печати. Результаты: из 5532
записей, выявленных посредством поиска в базе данных, 87 статей были включены в
качественный синтез. Широкое внедрение 3D-печати в ото-хнс все еще находится на
начальной стадии. тем не менее, он все чаще используется во всех специализациях:
от предоперационного планирования до проектирования и изготовления
индивидуальных имплантатов и хирургических шаблонов. новое приложение, которое
считается очень ценным, - это его использование в качестве учебного пособия для
медицинского образования и хирургической подготовки. Выводы: по мере развития
технологий и стандартов обучения, а также по мере того, как здравоохранение
движется к персонализированной медицине, 3D-печать становится ключевой
технологией в уходе за пациентами в отоларингологии. Лечащие врачи и хирурги,
желающие быть в курсе этих разработок, выиграют от фундаментального понимания
принципов и применения этой технологии. ларингоскоп, 129:2045-2052,
2019.\cite{hong2019}
