%% File              : 1.tex
%% Author            : Igor V. Sementsov <ig.kuzm@gmail.com>
%% Date              : 05.08.2023
%% Last Modified Date: 05.08.2023
%% Last Modified By  : Igor V. Sementsov <ig.kuzm@gmail.com>

Применение кастомизированных имплантов в хирургии 

В настоящее время, в хирургии широко променяются различные катомизированные
импланты. 

Так, индивидуальные имплантаты для реконструкции черепно-лицевых дефектов приобрели
важное значение из-за лучших характеристик по сравнению с их обычными аналогами.
это связано с точной адаптацией к области имплантации, сокращением времени
хирургического вмешательства и лучшим косметическим эффектом. Применение
3D-моделирования в черепно-лицевой хирургии меняет подход хирургов к
планированию операций, а графические дизайнеры разрабатывают индивидуальные
имплантаты. Развитие производственных процессов и внедрение аддитивного
производства для прямого производства имплантатов устранили ограничения формы,
размера, внутренней структуры и механических свойств, что сделало возможным
изготовление имплантатов, соответствующих физическим и механическим требованиям
региона имплантации. В этой статье будут рассмотрены последние тенденции в
области 3D-моделирования и индивидуальных имплантатов в черепно-лицевой
реконструкции.\cite{pmid24987592}

Поднадкостничные имплантаты были внедрены в прошлом веке. Плохие клинические
результаты привели к постепенному отказу от этих имплантатов. недавно несколько
авторов предложили возродить поднадкостничные имплантаты в качестве альтернативы
регенеративным процедурам. Целью данного исследования было описание клинического
применения изготовленного по индивидуальному заказу поднадкостничного имплантата
для несъемной ортопедической реабилитации беззубой верхней челюсти. Были
отсканированы гипсовые модели верхней и нижней дуги, а также макет. Данные
цифровой визуализации и связи в медицине, полученные с помощью конусно-лучевой
компьютерной томографии, обрабатывались с помощью процедуры пороговой обработки.
Дизайн поднадкостничного имплантата был нарисован на стереолитографической
модели и отсканирован. Как только цифровой проект поднадкостничного имплантата
был завершен, его отправили в аддитивное производство. после операции пациент
находился под строгим наблюдением до 2 лет. результаты оценивались на основании
возникновения биологических и механических осложнений, послеоперационных
осложнений и выживаемости имплантатов. послеоперационных осложнений у пациентки
не было. за период наблюдения не возникло ни биологических, ни механических
осложнений. в конце исследования имплантат все еще функционировал. изготовленные
по индивидуальному заказу поднадкостничные имплантаты могут рассматриваться как
альтернатива регенеративным процедурам реабилитации тяжелой костной атрофии. в
будущем необходимы дальнейшие исследования для подтверждения положительного
результата.\cite{36345834}

Изготовленные на заказ трехфланцевые вертлужные кейджи, напечатанные на
3D-принтере, являются новой опцией в арсенале ревизионного хирурга
тазобедренного сустава. в этой обзорной статье описываются эта технология, ее
показания, хирургический метод, преимущества, недостатки, использование, текущая
опубликованная литература и будущие применения.\cite{26351112}

Введение: сегодня при ревизионном тотальном эндопротезировании тазобедренного
сустава (РТА) используются различные варианты реконструкции потери вертлужной
кости. Целью исследования было сравнение результатов использования стандартных
вертлужных имплантатов (sais) и индивидуальных вертлужных имплантатов (cmais)
при арта в случаях с обширной потерей вертлужной кости. методы: проведен
сравнительный анализ результатов 106 операций РТА, выполненных за период с
января 2013 г. по декабрь 2019 г. В 61 случае (57,5\%) применялся КМАИС. в 45
случаях (42,5\%) саи были имплантированы. Результаты: частота асептического
расшатывания вертлужного компонента после РТА при неконтролируемой потере
костного фонда вертлужной впадины (тип III-IV по классификации Gross и
Saleh) при использовании cmai была меньше, чем при использовании sai (2,4\% и
10,0). \%, соответственно). наиболее значимые различия в показателях
асептического расшатывания отмечены после имплантации cmai и sai при
несплошности таза с неконтролируемым костным дефектом (0\% и 60,0\%
соответственно; p<0,001). Заключение: идеальными показаниями к использованию
ЦМАИ являются неконтролируемые дефекты и несплошность таза с
неконтролируемой потерей костного материала (типы III-V по общей
классификации и классификации Салеха). обработка этих дефектов саем приводит
к более высокой частоте асептического расшатывания, требующего повторных
ревизий. требуется дальнейшее наблюдение для оценки эффективности
использования cmai и sai в долгосрочном периоде наблюдения.\cite{34598861}

Применение трехмерной (3D) печати или аддитивного производства в области
хирургии позвоночника продолжает расти в количестве и масштабах, особенно в
последние годы, поскольку усовершенствованные технологии производства и
использование стерилизуемых материалов позволили создать 3D-печатные имплантаты.
Хотя 3D-печать в хирургии позвоночника изначально ограничивалась использованием
в качестве наглядного пособия при предоперационном планировании сложных
патологий, в последнее время она стала использоваться для создания
интраоперационных направляющих и шаблонов винтов для конкретного пациента и все
чаще используется в хирургическом обучении и обучении. Поскольку индивидуальное
лечение и персонализированная медицина набирают популярность в медицине,
3D-печать предоставляет аналогичную возможность для хирургических полей,
особенно при создании настраиваемых имплантатов. 3D-печать — относительно новая
область в хирургии позвоночника, поэтому в ней отсутствуют долгосрочные данные о
клинических результатах и ​​экономической эффективности; однако
очевидные преимущества и, казалось бы, безграничные возможности применения этой
развивающейся технологии делают ее привлекательным вариантом для будущей
хирургии позвоночника.\cite{31624730}

Черепно-челюстно-лицевая реконструктивная хирургия является сложной областью.
во-первых, он направлен на восстановление основных функций, а во-вторых, на
сохранение черепно-лицевых анатомических особенностей, таких как симметрия и
гармония. Трехмерные (3D) печатные биомодели получили широкое распространение в
медицинских областях, обеспечивая тактильную обратную связь и превосходное
понимание зрительно-пространственных взаимоотношений между анатомическими
структурами. Черепно-челюстно-лицевая реконструктивная хирургия была одним из
первых направлений, внедривших в свою практику технологию 3D-печати.
биомоделирование использовалось при черепно-лицевой реконструкции травматических
повреждений, врожденных нарушений, удаления опухолей, ятрогенных повреждений
(например, декомпрессивной краниэктомии), ортогнатической хирургии и
имплантологии. Доказано, что 3D-печать улучшает и позволяет оптимизировать
предоперационное планирование, разработать инструменты интраоперационного
руководства, сократить время операции и значительно улучшить биофункциональный и
эстетический результат. эта технология также продемонстрировала большой
потенциал в улучшении обучения студентов-медиков и ординаторов хирургических
специальностей. Цель этого обзора — представить современное состояние технологии
3D-печати, а также ее практическое и инновационное применение, в частности, в
черепно-челюстно-лицевой реконструктивной хирургии, проиллюстрированное двумя
клиническими случаями успешного использования технологии 3D-печати.\cite{28523082}

Использование 3D-печати приобретает значительный успех во многих областях
медицины, включая хирургию. здесь была внедрена технология повышения уровня
анатомического понимания благодаря характеристикам, присущим 3D-печатным
моделям: это высокоточные и индивидуальные репродукции, полученные на основе
собственных радиологических изображений пациентов, и представляющие собой
твердые захватываемые объекты, позволяющие свободно манипулировать ими. часть
пользователя. возникающая в результате тактильная обратная связь значительно
помогает пониманию анатомических деталей, особенно пространственных отношений
между структурами. в этом отношении они оказались более эффективными, чем
традиционные 2D-изображения и виртуальные 3D-модели. на сегодняшний день все
большее число приложений успешно протестировано во многих хирургических
дисциплинах, расширяя диапазон возможных применений до предоперационного
планирования, консультирования пациентов, обучения студентов и ординаторов,
хирургической подготовки, интраоперационной навигации и других; в последние годы
3D-печать также применяется для создания хирургических инструментов и
воспроизведения анатомических частей, которые будут использоваться
соответственно в качестве шаблонов или направляющих для конкретных задач
хирургии и индивидуализированных имплантируемых материалов в реконструктивных
процедурах. Будущие ожидания касаются, с одной стороны, сокращения
производственных затрат и времени для дальнейшего повышения доступности
3D-печати, а с другой - разработки новых методов и материалов, подходящих для
3D-печати биологических структур, с помощью которых воссоздается архитектура и
функциональность реального человека. органы и ткани.\cite{30167991}
