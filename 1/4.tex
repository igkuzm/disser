% File              : 4.tex
% Author            : Igor V. Sementsov <ig.kuzm@gmail.com>
% Date              : 29.12.2023
% Last Modified Date: 29.12.2023
% Last Modified By  : Igor V. Sementsov <ig.kuzm@gmail.com>
применение 3Д-печати в хирургии ранений лица

В этом отчете описывается хирургическая реконструкция большого челюстно-лицевого
дефекта, вызванного огнестрельным ранением с близкого расстояния, у собаки с
использованием титанового имплантата, индивидуального для пациента (пси).
3-летний мальчик-волк-овчарка поступил в больницу с обширным дефектом лица
справа и обнажением правой носовой полости, вызванным огнестрельным ранением.
Рентгенологическое исследование выявило тяжелую потерю правой верхнечелюстной,
носовой и резцовой костей, множественные переломы левой и правой небных костей,
а также оскольчатый перелом правой нижней челюсти. первоначальная хирургическая
процедура включала компьютерную томографию (КТ) для трехмерной (3d) конструкции
имплантата. Лечение открытой раны проводили в течение 18 дней до тех пор, пока
свежая грануляционная ткань полностью не покрыла раневое ложе. Имплантат был
разработан в форме «ручного захвата», чтобы закрыть дефект, выровнять
множественные сломанные небные кости и обеспечить функцию защелкивания. В
окончательный дизайн были добавлены несколько отверстий, включая кортикальные
отверстия для винтов. имплантат был напечатан на титановом сплаве. Хирургическое
наложение титана пси выполнено через 19 дней после первичной операции. Для
реконструкции слизистой оболочки правой полости носа использован свободный
сублингвальный трансплантат слизистой оболочки. затем слизистую покрыли
коллагеновой оболочкой для укрепления структуры полости носа. Выполнили тупую
диссекцию слизисто-надкостничной оболочки твердого неба над небным отростком и
небными костями, мягких тканей над верхней челюстью и зафиксировали напечатанный
на 3D-принтере титановый имплантат в заранее запланированном положении. Дефект
мягких тканей лица был реконструирован, титановый psi закрыт кожным лоскутом
angularis oris. произошел частичный некроз лоскута в ростральной части, рану
удалось заживить вторым натяжением. расхождение лоскута в месте соединения
лоскута и слизистой надкостницы твердого неба произошло при обнажении имплантата
через 2 дня после операции. Многократные попытки закрыть дефект не увенчались
успехом, и владелец захотел прекратить лечение. здоровая гранулированная ткань
наблюдалась проксимальнее имплантата. Через 60 дней после применения титана пси
дефект больше не увеличивался в размерах и не проявлял каких-либо заметных
осложнений, связанных с дефектом, и собаку выписали. Через шесть месяцев после
операции собака оставалась активной, имела отличный аппетит, набирала вес и
демонстрировала приемлемую симметрию лица без увеличения обнажения имплантата
или каких-либо проблем, связанных с имплантатом.\cite{36439351}

Цель: разработать хирургический шаблон для установки черепно-лицевых имплантатов
для фиксации носового протеза. материалы и методы: планирование положения
имплантата получено с помощью программного обеспечения для виртуальной хирургии;
позиции были перенесены в программу компьютерного моделирования свободной формы
и использованы для проектирования хирургических шаблонов. система быстрого
прототипирования использовалась для 3D-печати шаблона, состоящего из трех
частей: шлема для поддержки остальных, стартового шаблона для разметки кожи
перед поднятием лоскута и хирургического шаблона для сверления кости. Оценка
точности между запланированным и установленным окончательным положением каждого
имплантата проводилась путем измерения наклона оси имплантата (угловое
отклонение) и положения вершины имплантата (отклонение на вершине). Результаты:
имплантат в области надпереносья отличался по углу наклона на 7,78°, тогда как
два имплантата в предчелюстной кости отличались на 1,86 и 4,55° соответственно.
значения отклонения на вершине имплантатов относительно запланированного
положения составили 1,17 мм для имплантата в переносье и 2,81 и 3,39 мм
соответственно для имплантата в верхнюю челюсть. Выводы: протокол,
представленный в этой статье, может представлять собой жизнеспособный способ
позиционирования черепно-лицевых имплантатов для поддержки носовых
протезов.\cite{21198902}

В настоящем отчете описывается планирование и хирургическое вмешательство, а
также подводные камни и ведение пациента с почти тотальным отрывным повреждением
нижней челюсти, который был реабилитирован с использованием трехмерной (3d)
лазерной печати титановой нижней челюсти. Лазерное спекание включает в себя
стирание слоев металлического порошка для воссоздания трехмерного
имплантируемого скелетного дефекта. процесс включает использование либо
зеркального отображения непораженной стороны, либо использование базы данных
архивных изображений здоровых людей. 25-летний мужчина поступил с огнестрельной
травмой, в результате которой у него практически полностью оторвало нижнюю
челюсть. Пациент получил современное лечение с использованием лазерной
3D-печатной нижней челюсти, которая для функциональности была соединена с
жевательными мышцами. внутренняя сторона титановой челюсти была заполнена
оскольчатыми переломами костей пациента в дополнение к забранному костному
трансплантату гребня подвздошной кости, который был покрыт оставшейся
периостальной тканью пациента. Имплантация почти полной нижней челюсти с
использованием 3D-лазерной печати — это быстрый и предсказуемый процесс, который
у некоторых пациентов может привести к превосходным эстетическим и
функциональным результатам. авторы полагают, что в будущем черепно-лицевая
реконструкция будет использовать эти методы для реконструкции костей
лица.\cite{28005765}
