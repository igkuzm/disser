%% File              : 3.tex
%% Author            : Igor V. Sementsov <ig.kuzm@gmail.com>
%% Date              : 05.08.2023
%% Last Modified Date: 29.12.2023
%% Last Modified By  : Igor V. Sementsov <ig.kuzm@gmail.com>

современные принципы лечения огнестельный ранений лица

Краткая информация: колото-резаные ранения головы и шеи осколками в значительной
степени опасны для жизни и требуют немедленной медицинской помощи. В настоящей
статье хирургические состояния при ранениях головы и шеи, связанных с военными
действиями, представлены в условиях военной хирургии с учетом показателей
летальности. материалы и методы: исследование было задумано как ретроспективное
клиническое исследование «случай-контроль», включающее в основном 179 ранений
головы и шеи, полученных в результате осколков во время гражданской войны в
Сирии. записи 2015-2019 годов были проанализированы по демографическим
характеристикам, типам травм, локализации повреждений, подходам пластической
хирургии и послеоперационным исходам. Результаты: механизм ранения всех ранений
был проникающего типа, который чаще всего был вторичным по отношению к взрывному
устройству, обрушению вследствие осколков взрыва, огнестрельного оружия или
гранаты. при этом 43 (24\%) раненых были солдатами, 136 (76\%) - гражданскими
лицами. Учитывая причину осколочного ранения, 83\% - взрыв, а 17\% -
огнестрельные ранения. у 32\% были переломы лица. так как наиболее часто
встречались переломы лица в области челюстно-скуловой кости (28,4\%), глазницы
(22\%) и зубов (18,5\%). Учитывая зоны поражения шеи, в наибольшей степени
пострадал регион-2. в третьей зоне был самый низкий уровень травматизма - 10\%.
у 89 (49\%) пациентов авторы отдали предпочтение первичной реконструкции, а у 15
— вторичной реконструкции (8\%). авторы использовали лоскут Лимберга у 24 (32\%)
пациентов, ротационный лоскут у 39 (52\%) пациентов и двудольный лоскут у 12
(16\%) пациентов. вывод: важнейшей причиной смертности были не разрушения,
потери тканей, вызванные осколочным ранением или опытом хирурга, а тяжелые
состояния сепсиса или множественные различные травмы при доставке на лечение с
большого расстояния из зоны боевых действий.\cite{32209930}

В этом исследовании анализируется объединенная база данных реестра травм театра
военных действий армии США по черепно-челюстно-лицевым (CMF) боевым травмам
(bi), полученным военнослужащими США в конфликте в Ираке и Афганистане, чтобы
описать тип, распространение и механизм травм. методы и материалы: с 19 октября
2001 г. по 12 декабря 2007 г. был запрошен реестр травм суставов на предмет
наличия cmf bi, введенного в базу данных с использованием кодов icd-9; данные
были собраны для би-солдат. Результаты: нами выявлено 7770 би. около 26\% имели
cmf bi. среди би 2014 г. было 4783 смф би (2,4 ранения на одного
военнослужащего). большинство би-смф были мужчинами (98\%). средний возраст
составил 26 лет. cmf bi по родам войск: армия 72\%, морская пехота 24\%, флот
2\% и военно-воздушные силы 1\%. проникающие повреждения мягких тканей и
переломы составили 58\% и 27\% соответственно; 76\% переломов были открытыми,
24\% повреждений мягких тканей отмечены как осложненные. частота переломов лица
составила 36\% нижней челюсти, 19\% верхней/скуловой кости, 14\% носа и 11\%
орбиты. остальные 20\%, не указанные иначе. первичным механизмом травмы были
взрывные устройства (84\%). Выводы: двадцать шесть процентов всех би приходились
на площадь смф. На cmf bi приходится непропорционально большое количество
ранений, наблюдаемых в Ираке и Афганистане, по сравнению с предыдущими
американскими войнами. Механизм КМП би предполагает использование взрывных
устройств 84\%.\cite{23739250}

Цель: охарактеризовать и описать черепно-челюстно-лицевые (СМФ) боевые травмы,
полученные военнослужащими США в ходе операций «Свобода Ирака» и «Несокрушимая
свобода». Пациенты и методы: с 19 октября 2001 г. по 11 декабря 2007 г. в реестр
травм суставов были запрошены данные о боевых травмах cmf. Идентификация
повреждений КМП проводилась с использованием кодов «Международной классификации
болезней девятого пересмотра, клинической модификации» и данных, собранных по
военнослужащим службы боевой травмы. внебоевые травмы, погибшие в бою и случаи
возвращения в строй были исключены. Результаты: боевые травмы cmf были
обнаружены у 2014 из 7770 американских военнослужащих, получивших ранения на
поле боя. из 2014 раненых военнослужащих пришлось 4783 смс ранений (2,4 ранения
на одного военнослужащего). частота боевых ранений КМП по родам войск - армия -
72\%; морская пехота - 24\%; военно-морской флот - 2\%; и ВВС - 1\%. частота
проникающих ранений мягких тканей и переломов составила 58\% и 27\%
соответственно. из переломов 76\% были открытыми. Локализация переломов лица была
нижней челюстью в 36\%, верхней челюстью/скуловой костью в 19\%, носом в 14\% и
орбитой в 11\%. остальные 20\% не были указаны иначе. основной механизм травмы
связан с применением взрывных устройств (84\%). Выводы: из раненых американских
военнослужащих 26\% имели травмы области ВМС в ходе операции «Свобода
Ирака»/операции «Конфликты за непреходящую свободу» в течение 6-летнего периода.
Нередко наблюдались множественные проникающие ранения мягких тканей и переломы,
вызванные взрывными устройствами. Повышенная выживаемость благодаря
бронежилетам, передовой боевой медицине и более широкому использованию взрывных
устройств, вероятно, связана с повышенным уровнем травм на поле боя.  нынешнее
использование кодов «Международной классификации болезней девятого пересмотра,
клинической модификации» с регистром травм суставов не позволяет
охарактеризовать тяжесть ран лица.\cite{20006147}

Предыстория: улучшенная броня и боевая медицина привели к лучшей выживаемости в
войнах в Ираке и Афганистане, чем в любых предыдущих. Было предложено увеличить
частоту и тяжесть черепно-челюстно-лицевых травм. Всесторонняя характеристика
характера травм, полученных в течение этого 10-летнего периода в
черепно-челюстно-лицевой области, необходима для улучшения нашего понимания этих
уникальных травм, оптимизации лечения этих пациентов и потенциального
направления стратегического развития защитных средств в будущем. Методы: с 19
октября 2001 г. по 27 марта 2011 г. был проведен опрос объединенного реестра
травм театра военных действий, охватывающего операции «Несокрушимая свобода» и
«Свобода Ирака» по боевым травмам черепно-челюстно-лицевой области, включая
демографические данные пациентов и механизм травмы. Травмы классифицировали по
видам (раны, переломы, ожоги, повреждения сосудов и нервов) с использованием
Международной классификации болезней 9-й ред. диагностические коды. Результаты:
за 10-летний период черепно-челюстно-лицевые боевые травмы головы и шеи
обнаружены у 42,2\% больных, эвакуированных из операционной. в этой области высок
преобладание множественных ранений и открытых переломов. Первичный механизм
травмы связан с применением взрывных устройств, за которым следует
баллистическая травма. Вывод: современные боевые действия, характеризующиеся
взрывными ранениями, приводят к более высокой, чем ранее сообщалось, частоте
ранений черепно-челюстно-лицевой области. уровень доказательности:
эпидемиологическое исследование, уровень iv.\cite{23192069}

этиология челюстно-лицевых переломов (mffs) варьируется в зависимости от
географического положения и плотности населения. Целью этого исследования было
проанализировать этиологию, характер и лечение МФФ. эпидемиологические
характеристики и методы лечения МФФ никогда не оценивались в Сомали. в
исследование были включены 45 пациентов, оперированных по поводу МФФ в больнице
третичного уровня в Сомали (2018-2019 гг.). Были оценены демографические данные
пациентов, причины переломов, типы, связанные с ними нелицевые травмы, методы
лечения и время госпитализации. наиболее частыми этиологическими факторами МФФ
были взрыв (24,4\%) и нападение (24,4\%), за ними следовали огнестрельное
ранение (22,2\%), спортивный несчастный случай (15,6\%), дорожно-транспортное
происшествие (11,1\%) и падение с высоты (2,2\%). \%) пациентов соответственно.
Основным местом повреждения была кость нижней челюсти (64,4\%), за ней следовали
носовая кость, верхняя челюсть, скуловая и орбитальная области. наиболее
распространенными нелицевыми травмами у МФФ были разрывы мягких тканей (37,8\%),
за которыми следовали перелом бедренной кости (6,7\%), перелом ключицы (4,4\%) и
перелом бедренной кости с травмой грудной клетки (2,2\%). Наиболее применяемым
методом лечения была открытая репозиция микропластины +/- межчелюстная фиксация
(77,8\%). В связи с размерами переломов нижней челюсти выполнен подвздошный
аутотрансплантат (6,7\%). средняя продолжительность пребывания в больнице
составила 11,8 ± 8,4 дня (диапазон 1–45 дней), а некоторым пациентам (15,6\%)
потребовалась интенсивная терапия из-за тяжелых травм. это будет первое
исследование, направленное на анализ этиологии, характера и лечения МФФ в
Сомали. Это исследование посвящено социальным аспектам Сомали и показывает, что
MFF развиваются в результате сильного межличностного насилия у молодого
человека.\cite{34351730}

Отличительные механизмы подрыва тяжелой артиллерии и самодельных взрывных
устройств приводят к взрыву и «разбрызгиванию» высокоэнергетических осколков
различной формы, размеров и характеристик. Сопутствующие комплексные травмы
средней зоны лица различаются как по тяжести, так и по сложности задействованных
анатомических структур. Проблемы лечения начинаются со спасения жизни, которое
осложняется нарушением проходимости дыхательных путей, тяжелым кровотечением и
уникальными травмами верхней челюсти, носа и носо-орбито-решетчатой кости.
Двадцать два пациента были отобраны из числа не поддающихся количественному
определению пациентов, у которых наблюдалось массивное поражение средней части
лица. осколочные ранения. Для сохранения архитектуры, заживления и функции
средней части лица успешно использовалась йодоформная паста на ленточных
марлевых тампонах. Йодоформная паста на ленточном марлевом компрессе служит
двойной цели: сохраняет форму и каркас раздробленной стенки верхнечелюстной
пазухи и укрепляет фрагменты в положении для заживления. он также действует как
влажная повязка на оголенные фрагменты костей, останавливает кровотечение и
обладает противомикробными свойствами при тяжелых рваных ранах. При полной или
частичной потере ткани носа успешная процедура заключается в окончательной
ранней стабилизации каркаса с использованием интраназального модифицированного
стента портексной трахеостомической трубки для сохранения внутренней формы
носовой пирамиды.\cite{27380580}

Цель: современная баллистика и высокоэнергетический взрыв обладают
незамеченными, новыми и значительными биофизическими и патофизиологическими
поражающими эффектами, уникальными по сравнению с гражданскими травмами.
Первичное воздействие ударной волны сжатого воздуха в результате взрыва приводит
к разорванным и раздавленным повреждениям яичной скорлупы верхней центральной
части лица (ucm). Высокоэнергетические осколки снарядов различной формы и
размера вызывают обширные разрушения и отличаются от автоматных пуль тем, что
вызывают передачу высокой энергии тканям за счет создания временной кавитации.
методы: был выбран двадцать один пациент с не поддающимися количественной оценке
военными ранениями. Неотложная помощь при ранениях лица, спасающих жизнь,
начинается с опасного для жизни кровотечения или воздушной недостаточности. В
этой статье описывается неотложное лечение медиального сухожилия кантальной
области (mct), межкантальной области и тяжелых ранений носа. Результаты:
использованная процедура дала хорошие результаты по сравнению с результатами
случаев, которые лечились только с применением классического подхода к переломам
гражданского лица. Вывод: лечение травм УКМ является наиболее трудным, поскольку
УКМ включает в себя эстетическую, физиологическую и анатомическую области лица.
Предлагаемый метод обеспечивает немедленную превосходную стабильность мягких
тканей, костей и хрящей и хорошо переносится в долгосрочной перспективе как
тканями, так и пациентом. В большинстве случаев жертвам оказывают помощь при
ограниченных ресурсах, недостаточной специализации, массивных травмах, во время
массовых жертв, и один хирург должен справиться со всем этим в течение короткого
периода времени. Травмы ЦСМ действительно вызывают беспокойство, поскольку эта
область является основой эстетики и функции лица.\cite{30728703}

Нынешнее вооружение обладает незамеченными новыми биодинамическими поражающими
эффектами. у многих пострадавших высокоскоростные крупные осколки панциря
привели к массивным «отрубленным» повреждениям твердых и мягких тканей нижней
челюсти, сопровождающимся выпадением языка. Проблемы управления начинаются со
спасения жизни, которое, возможно, осложняется нарушением проходимости
дыхательных путей, сильным кровотечением, массивной потерей нижней челюсти и
выпадением языка. следовательно, целью должно быть «ни один пациент не должен
умереть только от массивных повреждений тканей лица». то есть, если это
возможно, в подходящее время применяются разумные методы спасения жизней. После
стабилизации общего состояния хирургическое лечение массивной потери ткани
нижней челюсти должно начинаться с немедленной реконструкции утраченной ткани.
семнадцать случаев были выбраны из неопределенного числа пациентов, у которых
были массивные потери ткани нижней челюсти, у которых язык, как ни удивительно,
остался нетронутым. в этих случаях окончательная ранняя стабилизация каркаса
была достигнута путем соединения двух оставшихся неповрежденных сегментов нижней
челюсти. Успешная процедура заключается в использовании 2-миллиметровой спицы
Киршнера в форме подковы, перекрывающей разрыв нижнечелюстной дуги, которая
эффективно используется в качестве «каркаса» для восстановления мягких тканей.
Правильное первоначальное хирургическое лечение привело к защите разорванной
ткани, уменьшению последующих осложнений и уродств, предотвращению выпадения
языка и сохранению физиологических функций неповрежденных тканей. Поскольку
спектр травм продолжает развиваться, клиническая характеристика тяжести ран лица
нуждается в расширенной классификации, соответствующей массивным травмам лица.
предполагается, что он имеет следующие дескрипторы: взрыв, проникновение,
перфорация, отрыв и «отрубание» (bppac).\cite{22892293}

Цель: особые механизмы воздействия первичных взрывов привели к переходной эпохе
травм лица. Повреждение механизма имплозии является одним из них. Поражение по
имплозивному механизму приводит к повреждению, ограничивающемуся газосодержащими
структурами слухового прохода, околоносовых пазух, желудочно-кишечного тракта и
легких. во всем мире число жертв взрывных взрывов возросло и значительно
улучшилось. Результатом является рост смертности и заболеваемости, а также новые
виды травм, особенно в челюстно-лицевой области. таким образом, знания и опыт
управления ими должны распространяться коллегами по всему миру посредством
публикаций. материалы и методы: имплозия и мини-ревзрыв сжатых воздухоносных
пазух приводит к скелетному размозжению носо-орбитально-решетчатых,
верхнечелюстных пазух и костей носа. в условиях войны успешно применялись
разнообразные хирургические подходы. Оценка сопутствующих повреждений легких
и/или головного мозга является первоочередной задачей при любом опасном для
жизни взрывном ранении. В этой статье описаны биофизические результаты взрывных
повреждений средней трети лицевого скелета и связанных с ними травм, а также
подробно описаны меры по управлению и защите измельченного воздуха, содержащего
околоносовые пространства. Результаты: легкое, простое и быстрое лечение было
успешно использовано на измельченной, фрагментированной скелетной архитектуре
средней части лица без увеличения заболеваемости и избежания ненужной
хирургической травмы. Выводы: травмы одной из самых сложных эстетических,
физиологических и анатомических областей тела лучше всего лечить с учетом
биофизического воздействия имплозивного механизма на воздухосодержащие
пространства челюстно-лицевой области. внедрение новых методов лечения тяжелых
деструкций твердых и мягких тканей позволит снизить частоту осложнений и
продолжительность операции.\cite{20006161}

Мощные взрывы могут нанести жертвам множество различных типов травм, некоторые
из которых изначально могут быть скрытыми. летящие обломки и сильный ветер
обычно вызывают обычные тупые и проникающие травмы. травмы, вызванные только
взрывной волной, являются результатом сложного взаимодействия с живыми тканями.
Интерфейсы между тканями разной плотности или между тканями и захваченным
воздухом приводят к уникальным закономерностям повреждения органов. они требуют
от внебольничного персонала, врачей скорой помощи и хирургов-травматологов
специально искать доказательства этих внутренних повреждений у людей с
множественными травмами, корректировать методы лечения, чтобы избежать
обострения опасных для жизни проблем, вызванных самой взрывной волной, и
обеспечить соответствующие меры. расположение этих пациентов в ситуациях
возможного массового поражения. знание потенциальных механизмов травмы, ранних
признаков и симптомов, а также естественного течения этих проблем во многом
поможет лечению пациентов, пострадавших от взрывной волны.\cite{11385339}

Известно, что огнестрельные травмы вызывают тяжелую заболеваемость и смертность,
когда поражаются области лица. Лечение огнестрельных ранений лица включает
обеспечение проходимости дыхательных путей, остановку кровотечения, выявление
других повреждений и окончательное восстановление травматических деформаций
лица. Цель настоящего исследования состояла в том, чтобы сравнить клинические
результаты (инфицирование и несращение) открытой репозиции и внутренней фиксации
с закрытой репозицией и челюстно-нижнечелюстной фиксацией (cr-mmf) при лечении
огнестрельных ранений нижней челюсти. Материалы и методы. Данное исследование
проводилось на кафедре челюстно-лицевой хирургии Медицинского университета
Шахида Зульфикара Али Бхутто/Пакистанского института медицинских наук,
Исламабад, Пакистан. девяносто огнестрельных переломов нижней челюсти были
случайным образом распределены в две равные группы. в группе А 45 пациентов
лечили открытой репозицией и внутренней фиксацией, тогда как в группе Б 45
пациентам также проводили закрытую репозицию и челюстно-нижнечелюстную фиксацию.
послеоперационные осложнения (инфекция, несращение) оценивались клинически и
рентгенологически в обеих группах. Результаты: у пациентов, получавших открытую
репозицию и внутреннюю фиксацию, наблюдалось больше осложнений с точки зрения
инфекции (17,8\%) по сравнению с закрытой репозицией (4,4\%) с значением p
0,044.  тогда как несращение наблюдалось чаще при закрытой репозиции (15,6\%) по
сравнению с группой открытой репозиции и внутренней фиксации (2,2\%) со значимым
значением p 0,026. Вывод: оба метода лечения могут быть использованы при лечении
огнестрельных ранений нижней челюсти, и необходимы дальнейшие исследования,
чтобы иметь четкие рекомендации в этом отношении в интересах пациентов,
сообщества и медицинских работников.\cite{32467805}

Челюстно-лицевая баллистическая травма представляет собой разрушительную
функциональную и эстетическую травму. обширное повреждение мягких и твердых
тканей непредсказуемо, и из-за разнообразия и сложности этих травм необходим
систематический алгоритм. В этом исследовании предпринята попытка определить
наилучший метод лечения челюстно-лицевых баллистических травм и описать
стандартизированный протокол хирургической и ортопедической реабилитации от
первой неотложной стадии до полной эстетической и функциональной реабилитации.
при низкоскоростных баллистических ранениях (скорость пули <600 м/с) рана обычно
менее серьезна и не смертельна, и лечение должно основываться на раннем и
окончательном хирургическом вмешательстве, связанном с реконструкцией, с
последующей реабилитацией полости рта. высокоскоростные баллистические
повреждения (скорость пули >600 м/с) сопровождаются обширным разрушением твердых
и мягких тканей, поэтому лечение должно основываться на трехэтапном
реконструктивном алгоритме: санация и фиксация, реконструкция и окончательная
ревизия. Реабилитация пациента с баллистической травмой — сложная многоэтапная
лечебная процедура, требующая длительного времени и многопрофильной команды для
обеспечения успешных результатов. Результат ортопедического лечения является
одним из важнейших параметров, по которым пациент оценивает восстановление
эстетических, функциональных и психологических недостатков. Данное исследование
представляет собой ретроспективный обзор: из базы данных отделения были выявлены
двадцать два пациента с диагнозом «исходы баллистических травм», одиннадцать
пациентов соответствовали критериям включения и были включены в
исследование.\cite{35743719}

Введение: Оскольчатые переломы нижней челюсти, вызванные огнестрельными
ранениями, традиционно лечились закрытой репозицией с использованием
челюстно-нижнечелюстной фиксации (MMF).^2,3 Открытая репозиция и внутренняя
фиксация (orif) стали ценным методом лечения оскольчатой нижней
челюсти. переломы из-за низкой частоты осложнений и предсказуемого заживления
^4, 5. Цель: сравнить эффективность orif по сравнению с mmf в достижении
костного сращения оскольчатых переломов нижней челюсти у пациентов с
огнестрельными ранениями. Метод:ология: рандомизированное контролируемое
исследование, проведенное в отделении челюстно-лицевой хирургии больницы Аббаси
Шахид в течение 3 лет; Всего 40 пациентов были разделены поровну на две группы.
группу а обрабатывали орифом, а группу b обрабатывали ммф. Образование костной
мозоли рентгенологически было подтверждено на 8-й неделе после операции. данные
были собраны с использованием формы, введенной в статистическое программное
обеспечение spss версии 20. Проценты частоты рассчитывались для возраста и пола.
Были применены хи-квадрат и точные тесты Фишера. Значение p ≤ 0,05 считается
значимым. Результат: в исследование были включены 40 пациентов с огнестрельными
ранениями. 37 (92,5\%) мужчин и 3 (7,5\%) женщин со средним возрастом 36,35 ±
12,9 лет (стандартное отклонение). У 19 (47,5\%) пациентов наблюдалось
образование костной мозоли, а у 21 (52,5\%) — нет. из 19 больных 14 (70\%)
относились к группе А, а 5 (25\%) — к группе Б. окончательное заживление,
рассматриваемое к 8-й неделе, было у 16 (80\%) пациентов из группы А и у 8
(40\%) группы В (MMF) после подсчета клинических и рентгенологических данных.
Вывод: сравнительные клинические исследования доказали, что orif превосходит mmf
при лечении оскольчатых переломов нижней челюсти. раннее первичное
восстановление и внутренняя фиксация обеспечивают предсказуемые и экономически
эффективные результаты.\cite{31806484}

Вводные данные: огнестрельные повреждения висцерокрана отмечаются редко.
Проникающие ранения черепно-челюстно-лицевой области представляют собой
серьезную проблему для хирургов, поскольку зачастую содержат серьезные дефекты
мягких тканей, костей и головного мозга. представляем историю болезни 42-летней
женщины с огнестрельным ранением висцерокрана после попытки суицида. имеются
серии изображений течения заболевания. Описание случая: женщина, 42 лет,
поступила с огнестрельным ранением висцерокрана после попытки суицида. при
поступлении показатель gcs составил 8/15, при общем осмотре входное отверстие
было в подбородочной области, а выходное - в левой височной области черепа.
после первых спасательных процедур перед хирургическим вмешательством
контролировали внутричерепное давление. В связи с массивным кровотечением
произведена эмболизация передней верхнечелюстной артерии. в дальнейшем были
выполнены трахеостомия, хирургическая репозиция множественных переломов
челюстно-лицевой области и отрыв глаза. во второй раз внутричерепное
мониторирование и компьютерная томография выявили признаки интрапаренхиматозного
повреждения. пациенту была проведена вторая хирургическая процедура, состоящая
из бифронтальной декомпрессивной краниэктомии. Больной выписан на 20-е сутки
после операции в реабилитационный центр. она вернулась в наше отделение через 4
месяца для выполнения черепно-челюстно-лицевой реконструкции. у нее было 15
гектаров, птоз слева, дефицит VII черепно-мозгового нерва слева, деканюляция,
КПС 100\%. Вывод: поэтапный мультидисциплинарный подход как с нейрохирургами, так
и с челюстно-лицевыми хирургами обязателен при огнестрельных ранениях черепа,
когда обширные повреждения связаны с более высокой смертностью.\cite{33295298}
