%% File              : 2.tex
%% Author            : Igor V. Sementsov <ig.kuzm@gmail.com>
%% Date              : 05.08.2023
%% Last Modified Date: 11.03.2024
%% Last Modified By  : Igor V. Sementsov <ig.kuzm@gmail.com>

Применение индивидуальных 3D-моделей в челюстно-лицевой хирургии

Введение: Кажется, что 3D-печать находит все больше и больше применений в
челюстно-лицевой хирургии (ЧЛХ), особенно с момента появления на рынке
3D-принтеров общего использования несколько лет назад. Целью нашего исследования
было ответить на 4 вопроса: 1. кто использует 3д печать в мфс и рутинно это или
нет? 2. Каковы основные клинические показания к 3D-печати в МФС и какие объекты
используются? 3. Эти объекты напечатаны официальным производителем медицинского
оборудования (МД) или изготовлены непосредственно в отделении или лаборатории?
4. Каковы преимущества и недостатки? методология: 1 января 2017 года в
паб-медике проведены два библиографических исследования без ограничения по
времени с использованием «челюстно-лицевой хирургии» и «3D-печати» для первого и
для второго «челюстно-лицевой хирургии» и «компьютерного проектирования» и
«компьютерное производство» в качестве ключевых слов. Были выбраны статьи на
английском или французском языке, посвященные клиническому использованию
3D-печати на людях. Были записаны дата публикации, национальность авторов,
количество пролеченных пациентов, клинические показания, тип печатного
объекта(ов), тип печати (лабораторная/больничная или
профессиональная/промышленная), а также преимущества/недостатки. Результаты:
критериям соответствовали двести девяносто семь статей из 35 стран. наиболее
представленной страной стала Китайская Народная Республика (16\% статей). в общей
сложности 2889 пациентов (в среднем 10 на статью) получили пользу от
3D-печатных объектов. наиболее частыми клиническими показаниями были
дентальная имплантация и реконструкция нижней челюсти. наиболее часто
печатаемыми объектами были хирургические руководства и анатомические модели.
сорок пять процентов отпечатков были профессиональными. основными
преимуществами были повышение точности и сокращение времени хирургического
вмешательства. Основными недостатками были стоимость предметов и срок
изготовления при печати в промышленности. обсуждение: появление на рынке
недорогих принтеров увеличило использование 3D-печати в мфс. анатомические
модели не считаются MDS и не должны подчиняться каким-либо правилам. в
настоящее время их легко распечатать на недорогих принтерах. они позволяют
лучше предоперационное планирование и подготовку к процедурам, а также
предварительному формированию пластин. окклюзионные каппы и хирургические
шаблоны предназначены для плавного переноса планирования в операционную. они
считаются mds, и даже если их легко распечатать, они должны соблюдать правила,
применимые к mds. Имплантаты для конкретного пациента (изготовленные на заказ
пластины и модули для реконструкции скелета) являются гораздо более
требовательными объектами, и их производство в настоящее время остается в
руках промышленности. Основным ограничением больничной 3D-печати являются
строгие правила, применимые к MDS. Основными ограничениями профессиональной
3D-печати являются стоимость и время выполнения заказа. Объекты, напечатанные
на 3D-принтере, сегодня легко доступны в MFS. однако они никогда не заменят
навыков хирурга и их следует рассматривать только как полезные
инструменты.\cite{louvrier2017}

Введение: в настоящее время сочетание классических конструкций поднадкостничных
имплантатов с 3D-изображением и печатью позволяет сократить время лечения и
обеспечить поддержку несъемных протезов в тех случаях, когда другие методы не
дают удовлетворительных результатов. Целью данного исследования является
представление цифровой техники изготовления индивидуальных поднадкостничных
имплантатов и того, какие осложнения могут возникнуть после такого типа
операции. Методы: в исследование были включены шестнадцать пациентов, которым в
период с октября 2021 г. по февраль 2022 г. применялся изготовленный по
индивидуальному заказу титановый поднадкостничный имплантат DMLS. Всем пациентам
проводилась ортопантомография (ОП) и конусно-лучевая компьютерная томография
(КЛКТ). измерениями, принятыми во внимание в этом исследовании, были прилегание
и стабильность имплантатов, продолжительность операции, выживаемость
имплантатов, а также ранние и поздние осложнения. Результаты: прилегание
имплантатов было крайне удовлетворительным, средняя оценка 4 из 5. Средняя
продолжительность вмешательства составила 86,18 мин. в конце исследования один
имплантат был утерян из-за недостаточного прилегания и рецидивирующих
неизлечимых инфекций. одиннадцать имплантатов (69\%) были установлены на верхней
челюсти и пять (31\%) имплантатов — на нижней челюсти. Выводы: принимая это во
внимание, изготовленные по индивидуальному заказу титановые поднадкостничные
имплантаты dmls могут обеспечить удовлетворительную приживаемость
имплантатов и низкую частоту осложнений.\cite{alexandru2022}

В этой статье обобщается современное использование индивидуальных имплантатов в
челюстно-лицевой хирургии.\cite{huang2019}

Так, индивидуальные имплантаты для реконструкции черепно-лицевых дефектов приобрели
важное значение из-за лучших характеристик по сравнению с их обычными аналогами.
это связано с точной адаптацией к области имплантации, сокращением времени
хирургического вмешательства и лучшим косметическим эффектом. Применение
3D-моделирования в черепно-лицевой хирургии меняет подход хирургов к
планированию операций, а графические дизайнеры разрабатывают индивидуальные
имплантаты. Развитие производственных процессов и внедрение аддитивного
производства для прямого производства имплантатов устранили ограничения формы,
размера, внутренней структуры и механических свойств, что сделало возможным
изготовление имплантатов, соответствующих физическим и механическим требованиям
региона имплантации. В этой статье будут рассмотрены последние тенденции в
области 3D-моделирования и индивидуальных имплантатов в черепно-лицевой
реконструкции.\cite{parthasarathy2014}

Поднадкостничные имплантаты были внедрены в прошлом веке. Плохие клинические
результаты привели к постепенному отказу от этих имплантатов. недавно несколько
авторов предложили возродить поднадкостничные имплантаты в качестве альтернативы
регенеративным процедурам. Целью данного исследования было описание клинического
применения изготовленного по индивидуальному заказу поднадкостничного имплантата
для несъемной ортопедической реабилитации беззубой верхней челюсти. Были
отсканированы гипсовые модели верхней и нижней дуги, а также макет. Данные
цифровой визуализации и связи в медицине, полученные с помощью конусно-лучевой
компьютерной томографии, обрабатывались с помощью процедуры пороговой обработки.
Дизайн поднадкостничного имплантата был нарисован на стереолитографической
модели и отсканирован. Как только цифровой проект поднадкостничного имплантата
был завершен, его отправили в аддитивное производство. после операции пациент
находился под строгим наблюдением до 2 лет. результаты оценивались на основании
возникновения биологических и механических осложнений, послеоперационных
осложнений и выживаемости имплантатов. послеоперационных осложнений у пациентки
не было. за период наблюдения не возникло ни биологических, ни механических
осложнений. в конце исследования имплантат все еще функционировал. изготовленные
по индивидуальному заказу поднадкостничные имплантаты могут рассматриваться как
альтернатива регенеративным процедурам реабилитации тяжелой костной атрофии. в
будущем необходимы дальнейшие исследования для подтверждения положительного
результата.\cite{strappa2022}

Черепно-челюстно-лицевая реконструктивная хирургия является сложной областью.
во-первых, он направлен на восстановление основных функций, а во-вторых, на
сохранение черепно-лицевых анатомических особенностей, таких как симметрия и
гармония. Трехмерные (3D) печатные биомодели получили широкое распространение в
медицинских областях, обеспечивая тактильную обратную связь и превосходное
понимание зрительно-пространственных взаимоотношений между анатомическими
структурами. Черепно-челюстно-лицевая реконструктивная хирургия была одним из
первых направлений, внедривших в свою практику технологию 3D-печати.
биомоделирование использовалось при черепно-лицевой реконструкции травматических
повреждений, врожденных нарушений, удаления опухолей, ятрогенных повреждений
(например, декомпрессивной краниэктомии), ортогнатической хирургии и
имплантологии. Доказано, что 3D-печать улучшает и позволяет оптимизировать
предоперационное планирование, разработать инструменты интраоперационного
руководства, сократить время операции и значительно улучшить биофункциональный и
эстетический результат. эта технология также продемонстрировала большой
потенциал в улучшении обучения студентов-медиков и ординаторов хирургических
специальностей. Цель этого обзора — представить современное состояние технологии
3D-печати, а также ее практическое и инновационное применение, в частности, в
черепно-челюстно-лицевой реконструктивной хирургии, проиллюстрированное двумя
клиническими случаями успешного использования технологии 3D-печати.\cite{matias2017}


