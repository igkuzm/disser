неотложная помощь при осколочных, осколочных и пулевых ранениях центрального
среднелицевого комплекса

Современная баллистика и высокоэнергетический взрыв обладают незамеченными,
новыми и значительными биофизическими и патофизиологическими поражающими
эффектами, уникальными по сравнению с гражданскими травмами. Первичное
воздействие ударной волны сжатого воздуха в результате взрыва приводит к
разорванным и раздавленным повреждениям яичной скорлупы верхней центральной
части лица (ucm). высокоэнергетические осколки снарядов различной формы и
размера вызывают обширные разрушения и отличаются от автоматных пуль тем, что
вызывают высокую передачу энергии тканям за счет создания временной кавитации.
методами был выбран двадцать один пациент с не поддающимися количественному
измерению военными ранениями. Неотложная помощь при ранениях лица, спасающих
жизнь, начинается с опасного для жизни кровотечения или воздушной
недостаточности. 

В этой статье описывается неотложное лечение медиального
сухожилия кантальной области (mct), межкантальной области и тяжелых ранений
носа. Результаты. Используемая процедура дала хорошие результаты по сравнению с
результатами случаев, которые лечились только с применением классического
подхода к переломам гражданского лица. Заключение. Лечение травм УКМ является
наиболее трудным, поскольку УКМ включает в себя эстетическую, физиологическую и
анатомическую области лица. Предлагаемый метод обеспечивает немедленную
превосходную стабильность мягких тканей, костей и хрящей и хорошо переносится в
долгосрочной перспективе как тканями, так и пациентом. В большинстве случаев
жертвам оказывают помощь при ограниченных ресурсах, недостаточной специализации,
массивных травмах, во время массовых жертв, и один хирург должен справиться со
всем этим в течение короткого периода времени. Травмы ЦСМ действительно вызывают
беспокойство, поскольку эта область является основой эстетики и функции лица.

Несмотря на технологические достижения XXI века, вооруженные конфликты, к
сожалению, доминируют в жизни более чем 80 стран мира. По оценкам, 90\% жертв в
результате нынешних конфликтов и террористических взрывов составляют гражданские
лица, большинство из которых составляют женщины и дети, по сравнению с прошлым
веком, когда 90\% погибших были военнослужащими [1]. к сожалению, во всем мире
большинству гражданских хирургов не хватает опыта лечения массовых ранений и
массовых жертв, и они меньше всего интересуются новостями о вооруженных
конфликтах.

Анатомически УКМ состоит из лобно-носового верхнечелюстного отростка/медиального
края глазницы, носовой кости и верхней части носовой перегородки/решетчатой
перпендикулярной пластинки. Промежуточная центральная часть лица
состоит из медиальной части нижнего края глазницы, передней антральной стенки и
парагрушевидной опоры носовых костей, которые располагаются над носовым
отверстием между лобно-носовыми отростками верхней челюсти, образующими боковые
стенки носа, и медиальными краями глазницы. - совместно [2]. Травмы
носо-орбито-решетчатой кости (НОЭ) являются наиболее трудными для
лечения переломами лица. даже с использованием методов черепно-лицевой хирургии,
компьютерной томографии (КТ), миниатюрных устройств для фиксации пластин и
винтов, а также коллективного опыта многих выдающихся хирургов, безупречный и
сабритный подход. шукер

Результат лечения до сих пор неизвестен [3]. Переломы в результате гражданских
травм в этой области приводят к расширению телекантуса и смещению медиального
кантального сухожилия (mct), которое находится далеко от своего анатомического
положения. они также могут привести к сложным повреждениям решетчатой
пластинки, утечке спинномозговой жидкости (СМЖ) и травмам головного
мозга; таким образом, необходимо искать признаки утечки спинномозговой жидкости
[4–8]. тогда как травмы от огнестрельного оружия различаются по своей
клинической картине и зависят от вовлеченных анатомических структур. Очевидно,
это наблюдается при повреждении верхней части средней части лица, поскольку
фрагменты костей глазницы, носа и черепа и хрящ спутаны в тяжах мягких тканей, и
анатомические ориентиры не видны. немедленная реконструкция и установка каркаса
в течение разумного периода времени предотвратят дополнительные осложнения,
поскольку эта область является основой эстетики и функции лица. С технической
точки зрения биофизика и патофизиология патофизиологические повреждения от
взрыва, осколка снаряда и пули напрямую связаны с количеством энергии,
передаваемой тканям-мишеням, вызывающими различные повреждения тканей. По
сравнению с другими частями лица, средняя часть лица оказывается наиболее часто
повреждаемой зоной. взрывные травмы в результате конфликтов или террористических
взрывов, которые производят первичные ударные волны сжатого воздуха, генерируют
значительную передачу кинетической энергии, вызывая уникальный рисунок и
маскированные смертельные эффекты, создавая серьезные множественные повреждения,
особенно для содержащих воздух структур, таких как носовые пазухи и легкие.
Молекулы воздуха сжимаются до такой высокой плотности, что сама волна давления
действует больше как твердый объект, ударяющийся о поверхность ткани. Слотник
[9] заявил, что тонкий слой сжатого воздуха этой положительной взрывной волны
движется во всех направлениях, оказывая давление до 700 тонн на квадратный дюйм
на атмосферу, окружающую точку взрыва, со скоростью до 13 000 миль в час или
29900 кадров в секунду. Клинические данные показали, что имплозия и миниатюрный
повторный взрыв пазух сжатого воздуха приводили к измельчению, разорванию и
раздавливанию скелета яичной скорлупы ноэ, верхнечелюстных пазух и структур носа
[10] (рис. 1, 2).

Патофизиология баллистической травмы напрямую связана с количеством энергии,
передаваемой тканям-мишеням. передаваемая энергия определяется скоростью и
массой ракеты, что выражается уравнением: e = mv2/2. Взрывы осколков
боеголовки/СВД приводят к образованию брызг высокоэнергетических осколков
разного размера, а раны, которые они наносят, отличительный. Шукер (2016) [11]
заявил, что осколки наносят обширные сложные травмы и поражение мягких и твердых
тканей, что больше нет морфолого-анатомических особенностей гайморовых пазух,
носа и ноэ. большая часть боевых ранений носит проникающий характер и вызвана
осколками разрывных боеприпасов (70–80\%), а не пулями, выпущенными из боевого
стрелкового оружия [12]. этиология этих повреждений обусловлена
​​сочетанием структурного характера, передачи кинетической энергии и
ее рис. 1 тяжелая взрывная травма средней части лица, демонстрирующая
разорванные и раздавленные носовые структуры рис. 2 переднезадняя
рентгенограмма черепа, показывающая скелетную травму раздавленной яичной
скорлупы и отсутствие анатомических особенностей в верхней центральной части
средней части лица из-за первичного воздействия взрывной волны в фазе j
имплозии. челюстно-лицевой. оральная хирургия. (январь–март 2019 г.)
18(1):124–130 125 123

взаимодействие с тканью, которое зависит от ее массы, размеров, плотности, угла
и части ее полета, а также характеристик ткани. Передача ке пули и биофизика ее
ранящего действия отличаются от таковых при осколочных ранениях, хотя они оба
имеют одинаковую массу и одинаковую скорость удара. повышенное повреждение
тканей может возникнуть в результате падения недеформирующихся винтовочных пуль,
что увеличивает передачу энергии к ране, увеличивая тем самым ее размер, а также
вызывая пульсирующий кавитационный эффект [13]. Давление пули, создаваемое в
тканях в результате временного эффекта кавитации, должно быть подвергнуто
сомнению в этой области, поскольку эта анатомическая архитектура состоит из
полостей разного размера и множества воздушных синусов, как и в УКМ. отсутствие
громоздких мышц обычно смягчает кавитационные эффекты, создаваемые
высокоскоростными ракетами в этом регионе [14].

Эти пациенты лечились с ограниченными медицинскими ресурсами в рискованных
условиях, с множественными травмами, массивными ранами, массовыми жертвами и
нехваткой узких специалистов. Двадцать один пациент с взрывными и
баллистическими комплексными травмами средней зоны лица был выбран из
неисчислимого большого числа. а. шесть пациентов получили травмы от взрывной
волны. четверо из них получили тяжелые раны с раздробленными и раздробленными
тканями в области УКМ. один из них получил тяжелую травму от размозжения яичной
скорлупы с небольшими рваными ранами на коже. один пациент был осмотрен через 3
недели по поводу отсроченного вторичного лечения взрывных повреждений средней
зоны лица, что потенциально приводит к значительным функциональным и
косметическим осложнениям, а также затрудняет позднюю реконструкцию для
участвующих хирургов. б. девять пациентов получили ранения осколками снарядов с
высоким и низким энерговыделением. у двух пациентов были проникающие осколки, а
у семи пациентов перфорационные повреждения привели к потере нежной
анатомической ткани, оставив после выхода бреши потерянной ткани. крупные
фрагменты вызывали уплощение переносицы, телекантус, энофтальм, птоз, подтекание
спинномозговой жидкости вследствие ринореи, внутриносовые разрывы, а также
септические и скелетные отрывы. в. три пациента получили огнестрельные ранения в
подбородочную область, проходящие через дно рта, языка и сломанную нижнюю
челюсть, проходящие через твердое небо, полость носа и область носа, что привело
к сложным травмам через него и выходу через лобную кость, как показано на
рисунке. инжир. 3, 4. Методы экстренной помощи при ранениях, полученных в
результате боевых действий или террористических актов, следует начинать с
определения приоритетности оценки любого опасного для жизни кровотечения или
нарушения воздушной среды. эти травмы сильно загрязнены и состоят из сильно
разорванных тканей; таким образом, ждать, пока гематома и отек спадут, в
«особенно жарком климате» не стоит. при этих повреждениях вторичные ткани рис. 3
вход винтовочной пули в подбородочную область и выход из лобной области. нанес
перелом тела нижней челюсти, вырвал язык, провел через небо и раздробил
интраназальную полость, повредив нос и выйдя через лоб рис. 4 отличных
результата при одном методе немедленной операции, фото пациента через 1 год

повреждение в зоне экстравазации является результатом ранней ишемии и
сосудистого поражения, вызванных высокоэнергетическим баллистическим
проникновением. Ввиду этого удаление мертвых или нежизнеспособных тканей
остается ключевым компонентом лечения ран, обеспечивая обычное обоснование
хирургической обработки. задержка этого действия более чем на 24 часа,
по-видимому, осложняет дальнейшее лечение, поскольку это связано с усилением
некроза тканей. Хирургическое лечение Неотложная помощь должна начинаться с
общего осмотра с головы до ног в течение 4–5 минут, во время которого могут
выявиться опасные для жизни состояния, такие как скрытое попадание снаряда в
грудную клетку или живот. Перед исследованием раны следует осторожно оценить
медиальную кантальную связку, межкантальное пространство, стенки носовой и
верхнечелюстной пазух. медиальная глазная связка. Хирургический подход
начинается с непосредственного лечения медиальной глазной связки и реконструкции
фрагментированной кости и мягких тканей ноя. в большинстве случаев сухожилие
требует минимально инвазивного доступа к медиальной области кантальной области
для прямой трансназальной фиксации с помощью проводов и изменения положения mct.
Двусторонний шов или шов со свободной иглой применяют для наложения швов по обе
стороны слезного сухожилия. это гарантирует, что mct полностью захвачен,
используется тонкая проволока, такая как стальная проволока калибра 26–30
(используется для двусторонней кантопексии), та же процедура выполняется на
контрлатеральной стороне и эти два шва завязываются. друг к другу, чтобы
привести двусторонний МКТ в правильное положение [15]. В случаях отсроченного
вторичного лечения при баллистических/взрывных ранениях хирургический подход
начинается с прорыва существующего рубца или раны в этой области. Что касается
выброса дна орбиты и запаивания глаза, хирургическое лечение не так просто, как
манипуляции и мобилизация новой раны. костный трансплантат из гребня подвздошной
кости был успешно использован для установки глазного яблока в правильное
положение, как показано на рис. 5, 6. межкантальная область экстраносально, в
межкантальной области, небольшим фрагментированным костям и тканям можно
придавать форму с помощью манипуляций, например, нажимая на них двумя пальцами
снаружи и элеватором изнутри, чтобы получить превосходный результат. форма
носовой области. нам следует быть осторожными, чтобы не затронуть решетчатую
область, толкая подъемник вверх по направлению к мозгу. пока двусторонние
медиальные стенки глазницы плотные, необходимо провести проволоку из нержавеющей
стали или нейлоновый шелк через зал двух пуговиц и трансназальную ткань с одной
стороны на другую, чтобы сохранить стабильность соответствующей морфологической
формы. интраназальный каркас, стентирование, шукер, 1988 г. [16] предложил новую
методику с использованием адаптированной портексной трахеостомической трубки,
подходящей для сильно поврежденной внутренней полости носа и сохраняющей
внутреннюю анатомическую форму и физиологическую функцию. это обеспечивает
стабильную проходимость дыхательных путей параболической формы в области
среднего прохода. выпуклый изгиб свода трубки нагревается в пламени, а верхняя
часть ее сдавливается прямой артерией рис. 5 отсроченная первичная
ударно-волновая травма: одностороннее размозжение ноя и области носа, а также
выдутое дно орбиты, запавший левый глаз и покрасневший ожог лица рис. 6 показаны
результаты реконструкции, начиная с фиксированного левого бокового костного
трансплантата дна орбиты. пациент через 5 месяцев с приемлемым результатом

щипцами и погружают в холодную воду для затвердевания. На трубки наращивают
твердые и мягкие ткани только с помощью наружных швов тканей кожи носа. Когда
медиальная стенка верхнечелюстной пазухи фрагментирована, внутри полости пазухи
следует использовать йодоформную пасту на упаковке из рибоновой марли, чтобы
сохранить первоначальную форму пазухи и предотвратить скопление тромбов. затем
трубки фиксируются друг к другу и к крылу носа. пуговицы можно удалить через 2
недели, в то время как интраназальные трубки остаются в таком положении в
течение месяца и более, как показано на рис. 7, 8, 9, 10 и 11.

Результаты. Полученные результаты заживления были приемлемыми функционально и
эстетически при использовании немедленного подхода. большинство пациентов имели
более короткий срок пребывания в больнице, более быструю реабилитацию и
требовали меньшего количества последующих обследований. посттравматические
утечки спинномозговой жидкости наблюдались редко и обычно со временем
разрешались. Немедленный неотложный хирургический подход защищает разорванные
ткани, останавливает кровотечение, уменьшает последующее обезображивание и
сохраняет эстетическую архитектуру области УКМ. этот метод применялся автором
регулярно в неисчислимом количестве во время иракско-иранской войны 1980–1988
годов, войны в Персидском заливе 1990 года и некоторых внутренних конфликтов. В
этих войнах широко применялись боеприпасы большой мощности; поэтому число
ежедневных жертв было очень высоким, к чему хирургам следует быть готовыми. этот
метод дал очень хорошие результаты по сравнению с результатами случаев, которые
лечились только с использованием стандартного подхода к гражданским переломам
(рис. 12). Хорошее понимание сложности биофизики и патофизиологического
взаимодействия взрывных, шрапнельных и винтовочных пуль, а также методов лечения
поможет медицинским работникам и клиницистам снизить смертность и заболеваемость
от травм. Хирургическую реконструкцию следует начинать с введения mct в качестве
центральной точки для комплекса noe как неотъемлемой части процесса, требующего
концентрации и внимания. межкантальную ткань можно стабилизировать с помощью
двух рис. 7 показан фрагмент панциря длиной 5–6 см, проникший на левую
латеральную часть носа и осевший в полости носа. наложена глубокая дистракция
тканей носа и носоэтмоидальной области глазницы рис. 8 истинная боковая
рентгенограмма черепа, на которой виден фрагмент панциря размером 5–6 см,
расположившийся в середине наружной полости носа и причинивший повреждение от
лобной пазухи до борозды верхней губы рис. 9 показана модифицированная
трахеостомическая трубка, введенная через носовую рану в качестве стента для
укрепления интраназальных фрагментированных тканей и носовой дуги 128 j.
челюстно-лицевой. оральная хирургия. (январь–март 2019 г.) 18(1):124–130 123
пуговицы для фиксации межкантального пространства, а затем фиксируются
нейлоновыми или шелковыми швами, проходящими через эту область. затем
использованная процедура интраназального стентирования обеспечила интраназальный
каркас и заживление фрагментированных тканей и достигла эстетичного
хирургического результата. методика переносится пациентом более одного месяца и
предотвращает коллапс носовой пирамиды и полный стеноз или блокаду полости носа
[16]. более агрессивные методы, используемые разными исследователями, но
сталкивающиеся со сложностью взрывных/баллистических повреждений средней части
лица. При современных боевых ранах проведение первичной реконструкции с
использованием фиксирующих костных отломков пластин и винтов, как правило,
нецелесообразно и требует много времени [17]. он резко лишает сегменты
оставшегося кровоснабжения, что приводит к последующей потере мягких тканей. это
приводит к отслоению слизистой оболочки от костных фрагментов и разрушению ткани
раны с инфицированием. нет возможности снять перчатки с корональной/средней зоны
лица или применить технику хирургического воздействия на открытом небе, которая
обеспечила бы наибольшее потенциальное раскрытие поверхности верхней и средней
областей лица при обычной травме [18, 19]. внешняя фиксация использовалась при
нестабильных «цепах» гражданских носовых костей, переломах и сильно смещенных
тканях носа. в этом случае для дорсальной поддержки используются
трансмукозальные и эндоназальные спицы Киршнера до тех пор, пока не произойдет
достаточное заживление [20]. Немедленная реконструкция медиальных стенок орбиты
и дорсальных носовых костных трансплантатов и аллотрансплантатов не
рекомендуется при неотложных военных ранениях, поскольку существует множество
факторов, ограничивающих ее применение. Я сталкивался со случаями, когда
медиальные канти смещались латерально. Я считаю, что проблему миграции кантала с
течением времени трудно предотвратить, поскольку заживление тонких
фрагментированных костных пластинок обычно происходит посредством фиброзного
сращения, независимо от используемой техники. Вывод: метод успешно применяется в
течение многих лет в качестве немедленной и минимально инвазивной хирургии,
которая сводит к минимуму риск поражения тканей, связанный с традиционным
подходом. Эта несложная и проверенная методика сводит к минимуму вероятность
заражения, сохраняет разорванную ткань и поддерживает оставшийся костный каркас
в правильном положении до тех пор, пока он не заживет. инжир. 10 извлеченных
крупных шрапнелей длиной 5–6 см, МКТ закреплены в правильном положении, 2
наружные пластиковые пластины (пуговицы) на межкантальном пространстве и 2
адаптированные портекс-трубки № 1. 8 для интраназальной стабильности. пациент
отлично выздоровел рис. 11 демонстрирует отличные результаты через 2 месяца
после операции. Рис. 12 осложнений боевых ранений носа демонстрируют полную
обструкцию носа, результат без использования интраназального стентирования и
каркаса из лацертидной ткани.

Благодарности Автор благодарит мисс Джиллиан Элдер и сотрудников библиотеки
Уэст-Блумфилд в Мичигане за помощь в написании этой статьи. финансирование нет
никаких финансовых интересов, которые могут иметь авторы в компаниях или других
организациях, которые заинтересованы в информации, содержащейся в статье
(например, национальные институты здравоохранения (nih); добро пожаловать
доверие; медицинский институт Говарда Хьюза (hhmi) (гранты, консультативные
советы, трудоустройство, консультации, контракты, гонорары, гонорары,
экспертные заключения, партнерства или владение акциями в областях, связанных с
медициной). соблюдение этических норм конфликт интересов автор заявляет об
отсутствии конфликта интересов.
