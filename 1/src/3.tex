Недавнее появление трехмерных
изображений КТ, МРТ, обычной рентгенографии и эхо-изображений улучшило
визуализацию сложных патологий, но лишено тактильных качеств. Объекты,
напечатанные на 3D-принтере, можно использовать для изучения сложных случаев,
отработки процедур и обучения студентов и пациентов. [1]. кроме того, некоторые
современные хирургические процедуры сложны и требуют руководства, чтобы избежать
повреждения важных частей тела или получить приемлемый эстетический результат
[2]. в некоторых случаях это руководство требует значительного количества
ионизирующего излучения и может значительно увеличить время операции [3]. кроме
того, анатомические дефекты могут потребовать индивидуального протезирования для
максимально точного устранения повреждений [4]. Необходимость улучшения
визуализации и улучшения хирургических результатов привела к появлению
3D-печатных анатомических моделей, индивидуальных руководств для пациентов и
3D-печатных протезов. Растущее хирургическое применение 3D-печати сделало
интересным анализ текущего внедрения этой новой технологии. В этой статье
представлен обзор текущего использования методов 3D-печати в медицине человека,
в частности в хирургии, на основе систематического обзора литературы с
использованием трех основных литературных баз данных. мы попытались определить
области и области применения, в которых данная технология достаточно
распространена или использовалась несколько раз, а также сообщить о ее
потенциальных преимуществах и недостатках. Поскольку бюджеты здравоохранения
находятся под давлением, а больницы и врачи желают повысить эффективность, мы
включили в анализ стоимость и эффективность затрат в качестве переменных. В
результате возникли следующие исследовательские вопросы: (1) какие хирургические
применения 3D-печати обычно используются в медицине? (2) какие преимущества,
недостатки и финансовые последствия имеют хирургические применения 3D-печати по
сравнению со стандартами медицинской помощи? Методы: систематический обзор
литературы был проведен с использованием Web of Science, pub med и embase.
Стратегия поиска оставалась широкой, чтобы гарантировать, что ни одна
релевантная статья не будет исключена. поисковые заголовки были «3D-печать»,
«трехмерная печать», «аддитивное производство» и «быстрое прототипирование».
после консультации с экспертом был проведен дополнительный поиск, включающий
приложения для 3D-печати, называемые шаблонами и имплантатами, ориентированными
на конкретного пациента.

Также были добавлены соответствующие статьи, найденные в ссылках. первоначальный
поиск в базе данных был проведен в феврале 2015 г. дополнительный поиск был
проведен в декабре 2015 г., чтобы включить все статьи, опубликованные в 2015 г.
только полные документы контролируемых исследований и серии случаев минимум из
четырех случаев, написанные на английском языке, где применяется 3D-печать для
хирургических целей на живых людях. Ручная проверка названий и рефератов
проводилась таким образом, чтобы включать только статьи, соответствующие
применению методов 3D-печати в медицинских целях. Критериями включения было
использование «компьютерного производства» (CAM), «компьютерного проектирования»
(CAD), «аддитивного производства» (am), «печатных каркасов», «стереолитографии»
и «обратного проектирования» для медицины человека. . кроме того, были сохранены
названия, содержащие слова «индивидуально», «индивидуально для пациента»,
«шаблоны» и «физическая модель», чтобы не упустить из виду потенциальное
использование. страница 3 из 21 Tack et al. bio med eng on line (2016) 15:115
исключены примеры виртуального 3D-моделирования или рендеринга без физических
3D-моделей. рассматривались только клинические применения; Исследования на
трупах, in vitro и животных не сохранились. Были выбраны только серии случаев с
более чем тремя случаями и клиническими испытаниями, поскольку мы связываем их с
более высокой интеграцией технологии в медицинскую область. публикации,
написанные на языках, отличных от английского, или публикации без полной версии
статьи, были исключены на основании аннотации. документы, сохраненные после
полнотекстового обзора, были подробно проанализированы с использованием таблицы
доказательств, чтобы сообщить о соответствующих характеристиках и результатах
исследования. Основываясь на часто сообщаемых в литературе результатах, мы
включили следующие переменные: влияние на время операционной (или) или время
лечения, уровень точности напечатанной детали, влияние на воздействие радиации,
клинический результат, стоимость и стоимость. эффективность. Влияние на
время/время лечения относится к экономии времени в операционной или на самом
лечении по сравнению с традиционной процедурой. сюда не входит экономия на
реабилитации и не учитываются какие-либо дополнительные работы, выполняемые
хирургом перед операцией.

точность напечатанной детали использовалась для оценки качества напечатанной
детали. для анатомических моделей учитывалось сходство с исходной формой. для
шаблонов и имплантатов точность напечатанной части оценивалась на основе
интраоперационной адаптации и необходимости прервать запланированную процедуру в
пользу традиционной процедуры. появление небольшого количества изменений в
руководстве или несколько процедур, преобразованных в традиционную процедуру,
считалось отражающим хорошую точность. Радиационное воздействие было
зафиксировано, когда оно было прямо упомянуто авторами. клинический результат
оценивался как повышение точности хирургического вмешательства или улучшение
конечного результата. Обратите внимание, что существует совпадение между
точностью напечатанной части и клиническим результатом, поскольку точные шаблоны
приводят к лучшему послеоперационному выравниванию и, следовательно, к
положительному результату. стоимость была учтена при упоминании авторами.
поскольку некоторые авторы начали обсуждать экономическую эффективность, мы
учитывали эту переменную, когда она упоминалась. По результатам первичного
поиска по базе данных в феврале 2015 г. было отобрано 7482 статьи.
дополнительный поиск в декабре 2015 г., включающий все публикации 2015 г., дал
1114 статей. Удалено 3386 дубликатов. В результате проверки названий было
сохранено 1873 статьи, из которых 2223 статьи были исключены. Для полного
прочтения отобрано 353 доклада; Было исключено 1520 статей, большинство из
которых представляли собой тематические исследования. после прочтения полных
статей 224 статьи были сохранены для дальнейшего анализа. за исключением трех
статей, все они были хирургическими. нехирургические документы были исключены.
шесть соответствующих статей, найденных в ссылках на принятые статьи, были
добавлены в таблицу окончательного анализа, в результате чего общее количество
статей достигло 227. Обзор выбранных статей, ранжированных по медицинским
областям, приведен в дополнительном файле 1. одна статья была разделена в трех
случаях, поскольку три разных исследования были опубликованы вместе. другая
статья была разделена на две части, поскольку в ней обсуждались два разных
исследования. в результате в 227 включенных статьях было получено 230
наблюдений. стратегия поиска и причины исключения приведены на рис. 1. страница
4 из 21 Tack et al. bio med eng on line (2016) 15:115 только две статьи были
датированы до 2000 года. восемь статей были датированы между 2000 и 2005 годами,
30 — между 2006 и 2010 годами и 189 — с января 2011 года по 25 февраля 2015
года. На рисунке 2 представлен обзор количество отобранных статей в год.
опубликованные результаты по 3D-печати чаще всего касаются хирургических
шаблонов (60 \%) и моделей для планирования хирургического вмешательства 
(38,70\%) (рис. 3). кроме того, имеются сообщения о результатах использования 
3D-печати для изготовления индивидуальных имплантатов (12,17 \%), форм для 
протезирования (3,91 \%), моделей формирования имплантатов (1,74 \%) и моделей 
для подбора
пациентов (0,87 \%). Обратите внимание, что в некоторых статьях методы
3D-печати использовались для разных целей, в результате чего общий
результат превысил 100 \%.

Отчеты о результатах 3D-печати касаются нескольких хирургических областей.
Наибольшая доля приходится на ортопедию – 45,18 \% (рис. 4): сюда входят
ортопедия колена (30,70 \%), бедра (8,33 \%), плеча (2,19 \%) и кисти (1,75 \%). 
Большую долю составляет также челюстно-лицевая хирургия (24,12 \%). за этим 
следуют краниальная хирургия и хирургия позвоночника, составляющие 12,72 и
7,46\% соответственно. более подробные результаты собраны в обзорной таблице
(таблица 1). данные организованы с использованием технологии и дисциплины. В
каждой категории дан обзор количества статей. общее количество 270 превышает
общее количество статей, поскольку одна статья может охватывать несколько
применений 3D-печати. первая переменная в таблице — это влияние на время в
операционной (или)/время лечения. сокращение времени работы оценивается как
выгодное. во-вторых, оценивается точность печатной детали. как объяснялось выше,
радиационное воздействие учитывается только в том случае, если изменение
радиационного воздействия явно упомянуто в документе. Медицинский результат и
стоимость являются окончательными регулярными

Анатомические модели Анатомические модели можно использовать для формирования
имплантатов в челюстно-лицевой хирургии – тема, которая обсуждалась в девяти
исследованиях [33–41]. в пяти статьях сокращение времени упоминалось как
преимущество [33, 36, 38–40]. восемь исследований пришли к выводу, что
напечатанные модели обеспечивают хорошее анатомическое представление, а девять
исследований отметили улучшение хирургических результатов. в двух исследованиях
упоминалось воздействие ионизирующего излучения [36, 41] и в двух упоминалось
увеличение затрат [39, 41]. анатомические модели также используются при отборе
пациентов для сердечно-сосудистой хирургии; это обсуждалось в двух исследованиях
[42, 43]. ни в одной из статей не упоминалось о сокращении времени, воздействии
ионизирующего излучения или медицинских результатах. в одной статье было
обнаружено, что модель хорошо отражает реальную патологию, но не упоминаются
связанные с этим затраты [42]. в другой публикации упоминалось, что затраты
увеличились в результате использования анатомической модели [43].

во многих областях используются анатомические модели для хирургического
планирования. Наши исследования показали, что анатомические модели используются
в сердечно-сосудистой хирургии, сосудистой нейрохирургии, стоматологической
хирургии, общей хирургии, челюстно-лицевой хирургии, нейрохирургии,
черепно-орбитальной хирургии, ортопедии и хирургии позвоночника [1–3, 9, 14, 15,
35, 37, 37, 39, 43–121]. из 89 исследований в 48 (53,93 \%) отмечалось сокращение
времени, проведенного в операционной. в двух (2,24 \%) исследованиях
упоминалось увеличение времени в операционной, а в 37 (41,57 \%) не
упоминалось какое-либо влияние на время в операционной. только в 13 из 48
исследований упоминалось сокращение времени в операционной и подтверждалось
это утверждение фактическими цифрами или статистикой [3, 39, 44, 72, 74, 78,
81, 84, 99, 107, 117, 119, 120]. в 80 (89,89 \%) изданиях печатная часть
показал хорошую точность, хотя численно это было подтверждено только в четырех
исследованиях [3, 81, 97, 106]. воздействие ионизирующего излучения не
упоминалось в 77 (86,51 \%) публикациях, а в восьми упоминалось снижение
облучения [3, 59–61, 74, 79, 101, 107]. в трех публикациях упоминается
повышенное воздействие ионизирующего излучения [92, 111, 114]. ни в одной
публикации не упоминалось об уменьшении медицинских результатов при
использовании анатомических моделей, тогда как в 73 публикациях упоминалось об
улучшении медицинских результатов. Что касается затрат, то в 52 публикациях
затраты не упоминались, в четырех — снижение затрат, а в 32 — увеличение затрат.
две трети исследований, в которых сообщалось об увеличении затрат, подтверждали
это утверждение цифрами или статистикой. восемь исследований, из которых четыре
использовали модели для челюстно-лицевой хирургии, оценили экономическую
эффективность анатомических моделей [44, 58, 67, 74, 79–81, 97].

формы для протезирования. Методы 3D-печати могут быть использованы для
изготовления форм для изготовления протезов, как обсуждалось в трех
исследованиях [45, 122, 123]. с таким подходом мы столкнулись в краниальной
хирургии, челюстно-лицевой хирургии и хирургии уха. во всех исследованиях
напечатанные детали были точными и улучшали результаты лечения. оба краниальных
исследования обсуждались в одной статье. в одном из этих исследований сокращение
времени упоминалось как преимущество [45]. исследование с использованием
3D-печатных форм для протезирования ушей показало, что их использование снижает
затраты и является экономически эффективным [123]. ни в одном из этих
исследований не упоминалось воздействие ионизирующего излучения.

хирургические шаблоны являются наиболее популярным медицинским применением
3D-печати: они упоминаются в 137 из 270 статей (50,74 \%) [10, 15, 30, 31, 39,
48, 59, 60, 62, 70, 71 , 73, 74, 76, 77, 79–81, 83, 84, 86, 88, 89, 92, 93,
96–98, 106, 108, 111–113, 118, 124–226]. Помимо ортопедии (направляющие для
эндопротезирования коленного сустава), 3D-печатные хирургические шаблоны также
использовались в нейрохирургии, стоматологической хирургии, хирургии
позвоночника и челюстно-лицевой хирургии. 28 из 53 исследований, в которых
упоминалось сокращение времени в операционной, также подтвердили это утверждение
цифрами или статистикой [39, 74, 81, 84, 118, 131, 132, 135, 136, 140, 141, 145,
151, 152, 162, 175, 177, 181, 190, 194, 196, 200, 207, 210–212, 219]. Увеличение
процедурного времени наблюдалось в семи статьях, пять из которых подтверждали
это цифрами или статистикой [62, 73, 125, 143, 153, 161, 225]. В 88
исследованиях сообщалось, что руководства имели хорошую точность, в 23 - о
средней точности, а в десяти - о недостаточной точности. Интересно, что в шести
из десяти статей, сообщающих о недостаточной точности, это подтверждается
цифрами или статистикой [148, 165, 182, 185, 191, 211]. радиационное воздействие
не упоминалось в 123 (89,13 \%) исследованиях. меньшее количество радиации было
отмечено в девяти исследованиях, в том числе в шести из 11 исследований по
хирургии позвоночника. Хирургические шаблоны улучшили клинические результаты в
86 (62,31 \%) случаях, дали аналогичные результаты в 31 случае и оказали
негативное влияние на клинические результаты в семи исследованиях, все из
которых были ортопедическими. стоимость, связанная с руководствами, была
упомянута только в 42 исследованиях, из которых 39 заявили, что они более
дорогие, а в двух — что они одинаково дорогие. 19 из 39 исследований,
показавших, что новая технология дороже, подтвердили этот вывод цифрами или
статистикой. В десяти исследованиях было заявлено, что руководства были
экономически эффективными, а в шести — что они не были экономически
эффективными. ни одно из этих исследований не подкрепило эти утверждения
цифрами. Учитывая все варианты применения, новая технология 3D-печати сократила
время операционной в 46 \% исследований. В 76 \% исследований отмечалось, что
напечатанная часть имела хорошую точность, а в 72 \% отмечалось улучшение
медицинских результатов. с другой стороны, 33\% авторов заявили, что технология
дороже. В таблице 2 показано влияние медицинской 3D-печати на время работы в
операционной. Коррекция выбросов курсивом (выброс определяется как исследование
с сильно отличающимся результатом по сравнению со средним значением остальных
исследований в группе)

сокращение времени в операционной. Время в операционной всегда было одним из
основных аргументов в пользу медицинской 3D-печати. из 227 статей в 42
описывалось точное влияние использования технологии 3D-печати на время. в
большинстве случаев 3D-печать привела к экономии времени. результаты приведены в
таблице 2. 3D-приложения, такие как хирургические шаблоны для челюстно-лицевой
хирургии, модели для планирования хирургических операций на позвоночнике и
челюстно-лицевой области, а также модели для формирования имплантатов,
используемые в челюстно-лицевой хирургии, по-видимому, получают наибольшую
выгоду от этой технологии. На момент начала обсуждения этого обзора не
существовало другого анализа интеграции методов, области и использования
медицинской 3D-печати. примерно в середине 2015 г. Хаммад и др. рассмотрели 93
статьи, посвященные современным хирургическим применениям [227]. и их обзор, и
настоящий приходят к одинаковым выводам. этот обзор более подробный, в него
включены 227 хирургических статей и используется стандартизированная форма для
оценки этих статей. Одним из основных критериев включения было использование
материалов, напечатанных на 3D-принтере, в медицинских целях in vivo. Поэтому
документы, описывающие 3D-модели, используемые в целях медицинского обучения и
тестирования, не были включены. Были рассмотрены серии случаев из четырех или
более испытаний, поскольку мы считаем, что они отражают зрелость
технологического применения для конкретной области. Число публикаций,
соответствующих нашим критериям отбора, увеличивается: с 1999 г. было выбрано
только два исследования, тогда как в 2015 г. было 70 квалификационных
исследований, что свидетельствует о растущем интересе медицинского сектора к
технологиям 3D-печати. Детали, напечатанные на 3D-принтере, имеют несколько
целей в медицине. Хотя анатомические модели составляли наибольшую долю в первые
годы медицинской 3D-печати, заметно растущее значение 3D-напечатанных
направляющих. Хирургические шаблоны в настоящее время являются наиболее
распространенным типом применения 3D-печати: в 60\% исследований упоминается
использование печатных хирургических шаблонов. анатомические модели
Анатомические модели, напечатанные на 3D-принтере, широко используются в
хирургической сфере. наш обзор показывает, что в ортопедии их использование
оказалось полезным, особенно при комплексной замене тазобедренного сустава, где
единогласно сообщалось об улучшении медицинских результатов. Кроме того,
исследования черепных (в основном орбитальных) переломов показали улучшение
результатов, что связано с использованием анатомических моделей в качестве
ориентиров до и во время операции, чтобы лучше понять патологию и избежать
ошибок. эти черепно-анатомические модели часто также используются для придания
формы имплантату перед операцией, что приводит к улучшению прилегания
имплантата, улучшению медицинского результата и сокращению времени операции. Как
и в случае с анатомическими моделями, используемыми в ортопедических и
краниальных целях, наши исследования показывают, что модели позвоночника и
челюстно-лицевой области улучшают планирование операции и клинический результат,
одновременно сокращая время операции. кроме того, анатомические модели могут
снизить необходимость рентгеноскопии во время операций на позвоночнике, уменьшая
воздействие ионизирующего излучения. Наше исследование показало, что
анатомические модели полезны для планирования сосудистых процедур, таких как
чрескожная имплантация клапанов, восстановление аневризмы аорты и черепа, а
также хирургического планирования сложных врожденных пороков сердца. кроме того,
два сердечно-сосудистых исследования показали, что эти модели улучшают отбор
пациентов для эндоваскулярных процедур по сравнению со стандартной медицинской
визуализацией. страница 10 из 21 Tack et al. bio med eng on line (2016) 15:115
анатомические модели можно использовать непосредственно во время хирургических
процедур. Во время операции по пересадке зубов 3D-модели зубов используются для
подготовки донорского участка, что повышает вероятность успеха процедуры. кроме
того, анатомические модели рта используются для изготовления шаблонов для
сверления зубных имплантатов и изготовления индивидуальных обтураторов для
пациентов после максилэктомии. последнее сократило объем трудоемкой работы как
стоматологов, так и техников. кроме того, челюстно-лицевые модели часто
используются для придания формы имплантатам перед операцией, что еще больше
увеличивает скорость операции и одновременно улучшает клинические и эстетические
результаты. хотя анатомические модели могут использоваться сами по себе, в нашем
исследовании прослеживается тенденция к использованию анатомических моделей в
сочетании с печатными хирургическими руководствами. Помимо ранее упомянутых
преимуществ, анатомические модели можно использовать для обучения
студентов-медиков и улучшить общение с пациентами и их знания о патологии.
Хирургические шаблоны Наши исследования показывают, что хирургические шаблоны
широко используются в ортопедической хирургии, хирургии позвоночника,
челюстно-лицевой хирургии и стоматологической хирургии, причем более половины
выбранных исследований нашего обзора упоминают использование шаблонов. коленные
хирурги, кажется, больше всего заинтересованы в использовании шаблонов.
Уникально положительные результаты использования ортопедической бумаги для
коленного сустава в 2012 году уступили место более нейтральным результатам
спустя годы, что позволяет предположить, что первоначальное волнение утихло,
когда технология стала более распространенной. В более поздних исследованиях не
отмечается существенной разницы в клинических результатах между индивидуальными
рекомендациями и стандартными инструментами для тотального эндопротезирования
коленного сустава. повышенная сложность процедур и менее опытные хирурги с
небольшим объемом операций отдают предпочтение использованию хирургических
шаблонов. Помимо клинических результатов, руководства для конкретных пациентов
уменьшают количество необходимых хирургических лотков и немного сокращают время.
Согласно одной из выбранных статей, большее сокращение времени или времени
произошло тогда, когда хирурги стали больше привыкать к управляемой процедуре.
экономическую эффективность еще предстоит доказать, но недавние исследования, в
которых упоминается экономическая эффективность коленных направляющих,
показывают, что эта технология не предлагает достаточных преимуществ, чтобы
покрыть дополнительные затраты, связанные с направляющими. Согласно нашим
выводам, хирургические шаблоны сокращают время в операционной и улучшают
результаты операций на позвоночнике и черепе. это происходит за счет
моделирования на моделях и точного перевода предварительной операции с помощью
направляющих. более половины выбранных исследований сообщили о снижении
воздействия ионизирующего излучения (дополнительный файл 1) из-за снижения
потребности в рентгеноскопии. в челюстно-лицевой хирургии 3D-печатные модели и
хирургические шаблоны все чаще используются для реконструкций нижней челюсти и
ортогнатической хирургии. направляющие используются при резекции как
нижнечелюстной части, так и трансплантата, а также для реконструкции недостающей
части при онкологических резекциях и реконструкциях нижней челюсти. Согласно
результатам наших исследований, спинальные хирургические шаблоны точно
транслируют хирургическое планирование и делают результаты менее зависимыми от
опыта хирурга. аналогичные результаты наблюдаются при использовании шаблонов во
время стоматологических операций. некоторые авторы ставят под сомнение
систематическое использование стоматологических шаблонов из-за связанных с этим
более высоких затрат и предлагают использовать шаблоны только в сложных случаях.
наконец, стереотаксические приспособления, напечатанные на 3D-принтере, можно
использовать для имплантации глубоких

точность шаблона или модели и правильное размещение шаблона играют важную роль в
конечном клиническом результате или преимуществе, обеспечиваемом моделью.
поэтому совпадение между точностью и клиническим результатом неизбежно. Точность
шаблонов может варьироваться в зависимости от производителя, предоставившего
элемент, напечатанный на 3D-принтере, и времени между сканированием,
использованным для изготовления шаблона, и моментом операции. кроме того, для
обнаружения дефектных шаблонов необходим хирургический опыт. наконец,
использование МРТ или КТ влияет на точность руководства. Анатомические модели
индивидуальных имплантатов можно использовать в качестве форм для изготовления
протезов, как это видно из отдельных исследований черепной и ушной хирургии.
Кроме того, в операции по увеличению подбородка использовались индивидуальные
3D-печатные формы протезов, что привело как к сокращению времени операции, так и
к улучшению эстетического результата за счет соответствия индивидуальному
профилю. наконец, наше исследование (дополнительный файл 2) предполагает, что
методы 3D-печати могут успешно использоваться для непосредственной печати
окончательного имплантата, чаще всего в краниальной хирургии. Черепные
индивидуальные имплантаты, по-видимому, точны и требуют меньше времени, но при
этом связаны с улучшением клинических результатов почти во всех рассмотренных
исследованиях. Аналогичным образом, лотки и фиксирующие пластины, напечатанные
на 3D-принтере, улучшают медицинские результаты и сокращают время операционной
при челюстно-лицевой хирургии. более того, одно избранное исследование
продемонстрировало дополнительное преимущество улучшения костеобразования и
ангиогенеза при использовании индивидуальных имплантатов. наконец, полные зубные
протезы также можно изготавливать путем быстрого прототипирования. Результаты
различаются: в одном исследовании упоминается более низкая эстетика зубных
протезов, напечатанных на 3D-принтере, а в другом исследовании упоминается
эстетика, аналогичная стандартным зубным протезам, при этом подчеркиваются
преимущества моделирования лица перед печатью окончательного протеза. Общие
методы 3D-печати широко используются в медицинских целях. в большинстве
исследований, выбранных здесь, медицинский результат улучшается за счет
использования 3D-печати. однако мы считаем, что энтузиазм следует несколько
умерить, поскольку только 14 % исследованных исследований подтвердили это
утверждение цифрами, что делает это важное преимущество довольно субъективным.
Сокращение времени операции упоминается почти в половине выбранных исследований
и подтверждается цифрами только в двух третях этих случаев. в целом, большинство
приложений для 3D-печати сокращают время или, но между различными вариантами
использования можно увидеть большие различия. некоторые сокращения времени
слишком малы, чтобы привести к соответствующим выгодам. Хотя сокращение времени
является основным преимуществом, которое может способствовать значительному
сокращению финансовых средств, увеличение времени, необходимого для планирования
хирургического вмешательства, редко учитывается. в немногих исследованиях прямо
упоминалось об увеличении времени подготовки или обсуждался вопрос о возможности
аутсорсинга хирургического планирования. Согласно двум избранным исследованиям с
использованием хирургических шаблонов для артропластики коленного сустава,
хирурги и пациенты тратят больше времени на подготовку к операции, чем можно
сократить во время операции. кроме того, эти исследования показывают, что
планирование может быть более точным, если оно выполняется хирургом, чем при
аутсорсинге. хотя в подавляющем большинстве выбранных исследований не
упоминается воздействие ионизирующей радиации, две трети исследований, в которых
упоминается радиация, сообщают о снижении уровня заболеваемости на стр. 12 из 21
tack et al. bio med eng on line (2016) 15:115 это ионизирующее излучение. это
можно объяснить высокой долей операций на позвоночнике, в которых упоминалось
снижение воздействия ионизирующего излучения, поскольку рентгеноскопический
контроль является хорошо известной практикой в ​​​​этой
конкретной области. Было бы сомнительно экстраполировать этот вывод на другие
области, поскольку медицинская 3D-печать требует компьютерной томографии или
МРТ. первый из них подвергает пациента значительному количеству ионизирующего
излучения; С другой стороны, рентгеноскопический контроль используется не так
часто. пациенты могут получить дополнительную пользу от технологии, поскольку
анатомические модели улучшают понимание пациентами патологии и процедуры. это
приводит к улучшению взаимодействия между пациентом и врачом и повышению
удовлетворенности пациентов. Тактильные анатомические модели также могут помочь
студентам-медикам и хирургам улучшить свои знания. Экономическая эффективность
новой технологии предлагается в 7% выбранных публикаций, но нигде не
подтверждается цифрами. другие публикации ставят под сомнение экономическую
эффективность и приходят к выводу, что использование 3D-печати нерентабельно.
некоторые авторы отмечают, что сложность случаев может оправдать дополнительные
затраты на хирургические шаблоны. Растущее экономическое давление на
здравоохранение делает для исследователей все более важным рассмотрение
экономических сторон новых технологий и методов. даже небольшой анализ,
проведенный неэкономистами, может указывать на то, будет ли новая технология
экономически эффективной или нет. Для оценки приемлемости технологии потребуются
более полные исследования экономической эффективности как для сложных случаев,
так и для рутинных случаев с использованием 3D-печати. хотя это и было одним из
ключевых моментов данного обзора, в литературе можно найти мало данных по этому
поводу. Стоимость деталей, напечатанных на 3D-принтере, сильно зависит от
производственной мощности. дешевые настольные 3D-принтеры позволяют создавать
дешевые 3D-модели и руководства, но имеют меньше разрешений и контроля качества,
чем коммерческие производители, которые обязаны соответствовать высоким
стандартам качества. кроме того, заявленные затраты на детали, напечатанные
самостоятельно, различаются от автора к автору, при этом лишь немногие упоминают
прямые затраты на подготовку (КТ, МРТ, множественные отпечатки, программное
обеспечение и компьютер) или временные затраты, затраченные на разработку
модели. неоднородность этих напечатанных частей препятствует более глубокому
анализу. поэтому мы бы рекомендовали будущим исследованиям представить данные
гораздо более прозрачным и объективным образом и сделать первые шаги в расчетах
экономической эффективности. хотя мы рассмотрели дополнительные статьи,
найденные в ссылках выбранной публикации, мы понимаем, что некоторые
соответствующие статьи могли быть пропущены. мы включили серии случаев и
исследования с четырьмя или более наблюдениями, предполагая, что наиболее
интегрированные практики будут иметь публикации, в которых будет указано их
конкретное использование. это означает, что субъекты, о которых сообщалось
только в отчетах о случаях, могли быть пропущены, даже если они были хорошо
интегрированы. хирургические публикации рассматривались и анализировались с
использованием таблицы доказательств. не все аспекты, которые могут быть выгодны
для конкретного использования, могут быть учтены, особенно если эти преимущества
не являются прямым результатом 3D-печатной детали. Приложения медицинской
3D-печати, используемые только для тестирования, демонстраций и обучения, не
были включены в этот обзор. Вывод: 3D-печать уже хорошо интегрирована в
медицинскую практику и литературу. области применения варьируются от
анатомических моделей (в основном для хирургического планирования) до
хирургических шаблонов и имплантатов. основные преимущества, заявленные авторами
избранных статей, сводятся к минимуму

это ионизирующее излучение. это можно объяснить высокой долей операций на
позвоночнике, в которых упоминалось снижение воздействия ионизирующего
излучения, поскольку рентгеноскопический контроль является хорошо известной
практикой в ​​​​этой конкретной области. Было бы сомнительно
экстраполировать этот вывод на другие области, поскольку медицинская 3D-печать
требует компьютерной томографии или МРТ. первый из них подвергает пациента
значительному количеству ионизирующего излучения; С другой стороны,
рентгеноскопический контроль используется не так часто. пациенты могут получить
дополнительную пользу от технологии, поскольку анатомические модели улучшают
понимание пациентами патологии и процедуры. это приводит к улучшению
взаимодействия между пациентом и врачом и повышению удовлетворенности пациентов.
Тактильные анатомические модели также могут помочь студентам-медикам и хирургам
улучшить свои знания. Экономическая эффективность новой технологии предлагается
в 7% выбранных публикаций, но нигде не подтверждается цифрами. другие публикации
ставят под сомнение экономическую эффективность и приходят к выводу, что
использование 3D-печати нерентабельно. некоторые авторы отмечают, что сложность
случаев может оправдать дополнительные затраты на хирургические шаблоны.
Растущее экономическое давление на здравоохранение делает для исследователей все
более важным рассмотрение экономических сторон новых технологий и методов. даже
небольшой анализ, проведенный неэкономистами, может указывать на то, будет ли
новая технология экономически эффективной или нет. Для оценки приемлемости
технологии потребуются более полные исследования экономической эффективности как
для сложных случаев, так и для рутинных случаев с использованием 3D-печати. хотя
это и было одним из ключевых моментов данного обзора, в литературе можно найти
мало данных по этому поводу. Стоимость деталей, напечатанных на 3D-принтере,
сильно зависит от производственной мощности. дешевые настольные 3D-принтеры
позволяют создавать дешевые 3D-модели и руководства, но имеют меньше разрешений
и контроля качества, чем коммерческие производители, которые обязаны
соответствовать высоким стандартам качества. кроме того, заявленные затраты на
детали, напечатанные самостоятельно, различаются от автора к автору, при этом
лишь немногие упоминают прямые затраты на подготовку (КТ, МРТ, множественные
отпечатки, программное обеспечение и компьютер) или временные затраты,
затраченные на разработку модели. неоднородность этих напечатанных частей
препятствует более глубокому анализу. поэтому мы бы рекомендовали будущим
исследованиям представить данные гораздо более прозрачным и объективным образом
и сделать первые шаги в расчетах экономической эффективности. хотя мы
рассмотрели дополнительные статьи, найденные в ссылках выбранной публикации, мы
понимаем, что некоторые соответствующие статьи могли быть пропущены. мы включили
серии случаев и исследования с четырьмя или более наблюдениями, предполагая, что
наиболее интегрированные практики будут иметь публикации, в которых будет
указано их конкретное использование. это означает, что субъекты, о которых
сообщалось только в отчетах о случаях, могли быть пропущены, даже если они были
хорошо интегрированы. хирургические публикации рассматривались и анализировались
с использованием таблицы доказательств. не все аспекты, которые могут быть
выгодны для конкретного использования, могут быть учтены, особенно если эти
преимущества не являются прямым результатом 3D-печатной детали. Приложения
медицинской 3D-печати, используемые только для тестирования, демонстраций и
обучения, не были включены в этот обзор. Вывод: 3D-печать уже хорошо
интегрирована в медицинскую практику и литературу. области применения
варьируются от анатомических моделей (в основном для хирургического
планирования) до хирургических шаблонов и имплантатов. основные преимущества,
заявленные авторами избранных статей, заключаются в сокращении страницы 13 из 21
tack et al. bio med eng on line (2016) 15:115 время хирургического
вмешательства, улучшение медицинских результатов и снижение лучевой нагрузки. к
сожалению, субъективный характер и отсутствие доказательств, подтверждающих
большинство этих преимуществ, не позволяют сделать окончательные заявления.
Повышенная стоимость этой новой технологии и зачастую ограниченные или
недоказанные преимущества ставят под сомнение экономическую эффективность
3D-печати для всех пациентов и применений. несколько авторов указали, что
медицинская 3D-печать имеет большие преимущества при использовании в сложных
случаях и при участии менее опытных хирургов. сокращения: аддитивное
производство; CAD: компьютерное проектирование; кулачок: автоматизированное
производство; КТ: компьютерная томография; г-н: магнитный резонанс; или:
операционная; 3d: трехмерный. вклад авторов П.Т. провел основное исследование и
написал первоначальную статью. la участвовал в разработке исследования,
координировал исследовательский процесс и улучшал окончательную статью. Компания
jv участвовала в разработке исследования, координировала исследовательский
процесс и предоставила ценную информацию, которая способствовала написанию этой
статьи. pg участвовал в разработке исследования и улучшении рукописи. все авторы
прочитали и одобрили окончательную рукопись.
