
Черепно-челюстно-лицевая реконструктивная хирургия является сложной областью.
во-первых, он направлен на восстановление первичных функций, а во-вторых, на
сохранение черепно-лицевых анатомических особенностей, таких как симметрия и
гармония. Трехмерные (3D) печатные биомодели получили широкое распространение в
областях медицины, обеспечивая тактильную обратную связь и превосходное
понимание зрительно-пространственных взаимоотношений между анатомическими
структурами. Черепно-челюстно-лицевая реконструктивная хирургия была одним из
первых направлений, внедривших в свою практику технологию 3D-печати.
биомоделирование использовалось при черепно-лицевой реконструкции травматических
повреждений, врожденных нарушений, удаления опухолей, ятрогенных повреждений
(например, декомпрессивной краниэктомии), ортогнатической хирургии и
имплантологии. Доказано, что 3D-печать улучшает и позволяет оптимизировать
предоперационное планирование, разработать инструменты интраоперационного
руководства, сократить время операции и значительно улучшить биофункциональный и
эстетический результат. эта технология также продемонстрировала большой
потенциал в улучшении обучения студентов-медиков и ординаторов хирургических
специальностей. Целью данного обзора является представление текущего состояния
технологии 3D-печати, а также ее практических и инновационных применений, в
частности, в черепно-челюстно-лицевой реконструктивной хирургии,
проиллюстрированной двумя клиническими случаями успешного использования
технологии 3D-печати.

Реконструкция черепно-челюстно-лицевых дефектов, будь то вторичная травма,
резекция опухоли, ятрогенное поражение, инфекция, врожденные дефекты или
генетические нарушения, является особенно сложной и очень требовательной
областью. эти травмы серьезно влияют не только на основные функции пациента,
такие как зрение, дыхание, речь, жевание и глотание, но и на его внешний вид,
оказывая серьезное влияние на качество жизни и социальную роль. Хирургические
показания и подходы должны строго выбираться на основе ожидаемого хирургического
результата. восстановление оральных и черепно-лицевых функций является основной
целью, но никогда не забывая об эстетических чертах лица, стремясь к гармонии
лица и наиболее совершенной симметрии.1–3 В настоящее время аутологичные методы
реконструкции, а именно свободные лоскуты (малоберцовая кость и гребень
подвздошной кости), являются золотой стандарт для


черепно-челюстно-лицевая сложная костно-реконструктивная хирургия из-за
ограничений и болезненности региональных лоскутов (большая грудная мышца с
ребрами, трапециевидная мышца, височная мышца со сводом черепа), несмотря на их
преимущество в сопоставлении тканей, относительно конечного результата. однако
использование свободных лоскутов может быть ограничено наличием подходящих
донорских участков, особенно при больших дефектах, дальнейших дорогостоящих
операциях и проблемах с забором тканей; болезненность донорской области с
дополнительным дискомфортом для пациента; вероятность заражения как на
реципиентном, так и на донорском участке; и увеличение времени хирургического
вмешательства.1,3 таким образом, текущие исследования как биологических, так и
небиологических альтернатив продолжаются, при этом основной вклад вносят области
трехмерных (3D) печатных биомедицинских моделей. они обладают способностью
воспроизводить морфологию биологического организма.

структуру — процесс, называемый биомоделированием.3 3D-печать можно использовать
для печати имплантируемых протезов твердых и мягких тканей, хирургических
шаблонов для интраоперационного использования, биоклеточной печати
3D-тканей/органов или для создания каркасов для тканевой инженерии4–6. Особая
сложность черепно-лицевой анатомии, наличие или тесная связь с жизненно важными
структурами (нервами, кровеносными сосудами, мышцами, связками, хрящами,
костями, лимфатическими узлами и железами), уникальностью каждого дефекта и
вероятностью инфицирования требуют точное предоперационное планирование с
последующим высокоиндивидуализированным и тщательным выполнением плана.
передовые методы визуализации стали важным компонентом предоперационного
планирования в реконструктивной хирургии. Технология визуализации улучшает
предоперационное планирование, определяя пределы безопасности во время
абляционной хирургии опухоли.


материалов и методов
для исследования текущего состояния технологии 3D-печати и
ее клинического применения в хирургической области, особенно в
черепно-челюстно-лицевой реконструктивной хирургии, был проведен обзор последней
литературы с использованием pub med, Web of Science и других надежных
источников. В национальной медицинской библиотеке проводился систематический
поиск соответствующих статей на основе названия аннотации и с использованием
следующих ключевых слов: 3D-печать, быстрое прототипирование,
черепно-челюстно-лицевые дефекты, имплантаты, протезы и реконструктивная
хирургия. первоначально мы нашли 73 результата, которые были сужены до 54 после
проверки в соответствии с включением и
пример и описание предполагаемого реконструктивного результата.7 однако
традиционные методы, включая трехмерные реконструкции различных органов,
ограничены их представлением на двухмерных (2d) компьютерных мониторах, что не
позволяет физически взаимодействовать с моделью.5,8,9 3d Технология печати,
также известная как технология быстрого прототипирования, аддитивное
производство или технология твердой произвольной формы, не является новой
концепцией. Впервые описанный Чарльзом Халлом, он был внедрен в 1990-е годы в
области медицины путем создания физических моделей с помощью компьютерного
проектирования (САПР), но с 1980-х годов он стал использоваться в промышленном
дизайне9–12. В области биомедицины было признано, что несколько приложений
подходят для создания биомоделей для улучшения хирургического планирования и
моделирования в имплантологии, нейрохирургии и ортопедии, а также для
производства челюстно-лицевых протезов.2,13 Доступные 3D-объекты преодолевают
ограничения. 3D-визуализации, отображаемой на плоских экранах. Трехмерные
объекты могут быть созданы на основе объемных медицинских изображений
компьютерной томографии (КТ) или магнитно-резонансной томографии (МРТ).
Используя специальные алгоритмы постобработки, пространственную модель можно
извлечь из наборов данных изображений и экспортировать в машиночитаемые данные.
данные пространственной модели используются специальными принтерами для создания
окончательной модели быстрого прототипа. уменьшение размера 3D-принтеров до
доступных настольных 3D-принтеров), сокращение затрат времени и затрат,
позволило компьютерное 3D-моделирование 3D-биомоделей с использованием методов
3D-печати, таких как стереолитография (SL), многоструйное моделирование,
селективное лазерное спекание (SLS) струйная техника и моделирование
наплавленным осаждением (fdm).5,9 обеспечивая тактильную обратную связь и
превосходную оценку зрительно-пространственных взаимоотношений между
анатомическими структурами, использование этих 3D-моделей, потенциальные
преимущества огромны в различных областях медицины и черепно-челюстно-лицевой
реконструкции. В этой статье основное внимание уделяется практическим и
инновационным применениям, проиллюстрированным двумя клиническими случаями,
когда технология 3D-печати была успешно применена для черепно-челюстно-лицевой
реконструкции. как стимулировать дальнейший рост идей и развитие технологий в
хирургической области.


критерии исключения для формирования набора из 39 полнотекстовых статей,
отобранных для рассмотрения. Наши критерии включения включали оригинальные
статьи, написанные на английском языке не старше 15 лет, в которых ключевым
моментом были технология 3D-печати и ее практическое применение в
черепно-челюстно-лицевой реконструкции, а также будущие перспективы этой техники
и то, как она может влияют на повседневную хирургическую практику. Критерии
исключения включали исследования на животных, статьи не на английском языке и
тезисы конференций.

Результаты

3D-печатные биомодели, как упоминалось ранее, могут использоваться в качестве
точного инструмента тактильной визуализации и устройства хирургического
моделирования для воспроизведения сложных, уникальных для пациента патологий,
которые помогают хирургам предоперационно прогнозировать потенциальные
интраоперационные проблемы и послеоперационные последствия. приходит, а также
снижает риск осложнений.2,9 в зависимости от технологии производства также
возможно комбинировать в одной модели материалы разной эластичности или цвета,
что может быть полезно для создания более реалистичных моделей в образовательных
или исследовательских целях. или для естественно выглядящего имплантата. время
производства зависит от используемого метода, а также от размера и сложности
модели.8,16 Динамический процесс производства 3D-биомоделей, известный как
реверс-инжиниринг, состоит из следующих четырех основных последовательных
этапов1–3,10 ,12:

1. получение высококачественных объемных 3D-изображений анатомической структуры
с помощью КТ или МРТ для моделирования. Рекомендуется толщина среза КТ менее 1
мм. 
2. Обработка 3D-изображений для извлечения интересующей области из
окружающих тканей, для чего требуется два типа программного обеспечения:
во-первых, программное обеспечение для «3D-моделирования», которое преобразует
файлы цифровых изображений и коммуникаций в медицине (dicom) из
КТ/МРТ-сканирований в CAD-файл, выделяющий интересующую область, и, во-вторых,
программное обеспечение «3D-нарезки», которое делит CAD-файл на тонкие фрагменты
данных, подходящие для 3D-печати.17 
3. построение модели путем механической
обработки блока материала (субтрактивное производство). ) или чаще путем
послойного добавления материала и сплавления слоев (аддитивное производство).1
4. гарантия качества модели и ее точности размеров.

Крайне важно, чтобы биомедицинские модели, созданные с помощью технологии
3D-печати, подвергались строгому контролю качества на всех этапах
производственного процесса.15
                                                               
обзор технологий 3D-печати                                                               

Тип 3D-принтера, выбранного для конкретной задачи, часто зависит от доступного
материала и способа соединения слоев в готовом изделии. в целом точность
3D-принтера будет напрямую зависеть от точности компьютерной томографии,
особенно от того, толщина которой должна быть как можно меньшей.10

использование различных материалов в зависимости от желаемой степени прочности и
долговечности. основным ограничением является высокая стоимость этих принтеров,
что может ограничивать их более частое использование9.
Powder-Based 3D Printing Technology

селективное лазерное спекание

Технологию sls можно использовать для создания моделей из металла, пластика и
керамики в несколько этапов, что требует длительного времени изготовления и
высокой стоимости. сначала данные 2D-среза подаются в sls-аппарат, который
направляет луч лазера на тонкий слой порошка, такого как нейлон или такие
металлы, как титан, предварительно нанесенный на модельный лоток и выровненный
валиком. луч СО2-лазера нагревает частицы порошка, сплавляя их в

мы описали краткий обзор наиболее распространенных технологий 3D-печати,
используемых для черепно-челюстно-лицевой реконструкции, с указанием их основных
преимуществ и недостатков. Выбор метода зависит от интересующих материалов,
ограничений машины и конкретных требований окончательной трехмерной биомодели,
например клинического применения.

технология жидкостной 3D-печати
                                                                
стереолитография                                               
Стереолитография (SL) — наиболее широко используемый 3D-метод в черепно-лицевой
хирургии, при котором слой жидкого фотополимера или эпоксидной смолы на
платформе для построения модели отверждается ультрафиолетовым (УФ) лазером малой
мощности. зеркало, управляемое компьютером, используется для направления фокуса
ультрафиолетового лазера на поверхность смолы и отверждения смолы слой за слоем.
каждый из этих слоев соответствует эквиваленту среза аксиального изображения при
КТ/МРТ-сканировании.2,8 слои отверждаются последовательно и соединяются вместе,
образуя твердый объект, начиная с нижней части модели и поднимаясь вверх. .17–21
окончательная модель после снятия с опорных конструкций отверждается в
УФ-камере.9 в настоящее время SL считается золотым стандартом в производстве
3D-биомедицинских моделей, с лучшей гладкой поверхностью, быстрой обработкой, и
наибольшая точность (0,025 мм).4,9,22 однако этот метод также имеет несколько
недостатков — например, он требует обширной ручной обработки после производства
и высоких затрат, связанных с материалами, принтером и обслуживанием.17, Недавно
22 была разработана новая технология, называемая непрерывным производством
жидкостной границы (клипса), основанная на традиционном SL, но увеличивающая
скорость производства за счет использования кислородного ингибирования
ультрафиолетового излучения. однако этот метод еще не был оценен в
реконструктивной хирургии

полиструйное моделирование

Полиструйное моделирование (PM) выполняется путем распыления жидких
фотополимерных материалов ультратонкими слоями (16 мкм) на модельный лоток слой
за слоем до тех пор, пока модель не будет завершена. Преимущество этого метода
заключается в том, что каждый слой фотополимера немедленно отверждается
ультрафиолетовым светом сразу после его нанесения, что позволяет избежать
трудоемкой последующей обработки в УФ-камере.19 С помощью этого метода также
можно печатать


формируют сплошной слой, а затем перемещаются по осям X и Y для проектирования
конструкций в соответствии с данными CAD. после плавления первого слоя модельный
лоток перемещается вниз, наносится и спекается новый слой порошка, и процесс
повторяется до тех пор, пока модель не будет завершена, без поддержки во время
изготовления. поверхность прототипа непрозрачна, обычно абразивная и пористая и
обработана пескоструйной обработкой.19,20

3D-печать (биндерная струя)

система подачи связующего использует печатающую головку для выборочного
распределения связующего по слоям порошка. тонкий слой порошка распределяется по
лотку с помощью валика, аналогичного тому, который используется в системе sls.
печатающая головка сканирует лоток для порошка и подает непрерывную струю
раствора, который связывает частицы порошка при соприкосновении с ними.21
при изготовлении модели не требуются опорные конструкции, поскольку окружающий
порошок поддерживает несоединенные части. когда процесс завершен, окружающий
порошок аспирируется. он имеет меньшую стоимость и может быть использован для
формирования сложных геометрических структур.

технология 3D-печати на твердой основе
Моделирование плавленым осаждением (fdm) использует принцип, аналогичный sl, в
том, что модели строятся послойно. Основное отличие состоит в том, что слои
наносятся в виде термопласта, который выдавливается из тонкого сопла под
управлением компьютера. 3d-модель строится путем экструзии нагретого
термопластичного материала на поверхность пенопласта по пути, указанному данными
модели.19
по мере того как каждый слой пластика остывает, он затвердевает, постепенно
создавая окончательную модель, обладающую стабильностью, долговечностью и
механическими свойствами.2,12 В зависимости от сложности и стоимости
fdm-принтера он может иметь расширенные функции, такие как несколько печатающих
головок. принтеры FDM могут использовать различные пластмассы; наиболее
распространенным материалом, используемым для этой процедуры, является
акрилонитрилбутадиенстирол.22,24


клиническое применение 3D-печати

технологии в черепно-челюстно-лицевой области

реконструктивная хирургия

краниопластика
черепно-лицевые аномалии относятся к числу наиболее распространенных врожденных
дефектов/генетических нарушений у человека и имеют значительную
функциональные, эстетические и социальные последствия, которые требуют сложного
клинического и хирургического лечения.24 краниопластика является процедурой
выбора для коррекции дефектов черепа, обычно вызванных травмой, после резекции
опухоли, врожденными или врожденными дефектами или после декомпрессивной
краниотомии.1,2 до тех пор, пока В последнее время при реконструкции
черепно-лицевых дефектов точность моделирования недостающей части зависела
главным образом от навыков скульптора или хирурга.2 Биомоделирование изменило
эту реальность. возможность визуализировать модель под разными углами
способствовала интуитивному пониманию анатомических взаимосвязей между
структурами.9 создание индивидуальных имплантатов для реконструкции
черепно-челюстно-лицевых дефектов теперь стало реальностью с точной адаптацией к
области имплантации, что сокращает время и затраты на хирургическое
вмешательство, что, в свою очередь, приводит к уменьшению вероятности заражения,
более быстрому выздоровлению и улучшению эстетического результата.1 На основе
данных КТ создается трехмерная цифровая модель черепа. при первичных
реконструкциях эта модель используется для определения точных мест остеотомии и
разработки хирургического шаблона, который идеально соответствует дефекту. при
вторичных реконструкциях виртуальная модель затем используется для создания
дизайна имплантата либо путем зеркального отражения с контралатеральной стороны,
либо путем создания кривых на основе анатомической области с помощью устройств
на основе CAD.3 Идеальный материал черепного имплантата должен соответствовать
черепному дефекту и достигать полное закрытие (например, рентгенопрозрачное для
послеоперационной визуализации), устойчивое к инфекциям, устойчивое к
биомеханическим процессам, легко придаваемое, недорогое и готовое к
использованию.1,25–27


реконструкция верхней челюсти
Дефекты максиллэктомии становятся более сложными, когда вовлекаются критические
структуры, такие как орбита, глазное яблоко и основание черепа. реконструкцию
дефектов средней зоны лица можно разделить на следующие типы в соответствии с
модифицированной системой классификации максиллэктомий, предложенной Костой и
соавт.28:
• тип i: ограниченная максилэктомия • тип ia: исключение носочелюстной
структуры, с резекцией горизонтальной пластинки с сохранением верхнечелюстной
дуги 
• тип ib: включение носочелюстной структуры с сохранением передней части
верхнечелюстной дуги и неба 
• тип ic : включая носо-верхнечелюстные структуры с
резекцией передней верхнечелюстной дуги и неба 
• тип ii: субтотальная или
инфраструктурная максилэктомия 
• тип iia: с резекцией менее 50\% неба 
• тип IIb: с резекцией более 50\% неба небо 
• тип iii: тотальная максилэктомия • тип iiia: с сохранением содержимого глазницы 
• тип iiiam: с сохранением содержимого глазницы
						и резекцией нижней челюсти 
• тип iiib: с экзентерацией содержимого орбиты 
• тип iiibm: с экстракцией содержимого глазницы и резекция нижней челюсти 
• тип iv: орбитальная или супраструктурная максилэктомия 
• » тип iva: с сохранением крыши глазницы 
• » тип ivb: с резекцией крыши глазницы

В челюстной хирургии после визуализации и виртуальной реконструкции верхней
челюсти хирурги используют эту технологию для выполнения запланированной
цифровой остеотомии верхней челюсти и последующей реконструкции с использованием
гребня подвздошной кости или малоберцовой кости. цифровое отражение здоровой
верхней челюсти с аномальной стороны помогает добиться идеальной геометрии
скелета.1,3 в челюстно-лицевой хирургии это было успешно реализовано в виде
сменных приспособлений, которые ограничивают оперативное размещение определенной
областью. в результате запланированные остеотомии и движения костей точно
определяются и адаптируются к анатомии пациента.29 эти шаблоны широко
использовались в реконструктивной хирургии верхней и нижней челюсти, чтобы
заменить то, что ранее считалось золотым стандартом лечения малоберцовой кости.
костно-сосудистые свободные лоскуты.30–33.

Случай 1: максилэктомия типа IIb. Это случай 33-летней пациентки с диагнозом
хронический остеомиелит с поражением правой верхней челюсти. Ранее ее подвергали
множественным выскабливаниям, сеансам гипербарической оксигенации и различным
безуспешным испытаниям антибиотиков. Хирургическое планирование включало
виртуальное планирование (►рис. 1) на основе данных 3D-КТ (разрешение 1 мм) и
импортирование в файл CAD. таким образом, моделирование хирургической процедуры
с помощью 3D-биомодели (►рис. 2в), выполненной методом SL, позволило лучше
визуализировать скелетные структуры и создать индивидуальные хирургические
шаблоны для лучшего восстановления симметрии лица. ей была выполнена
правосторонняя субтотальная максилэктомия с последующей немедленной
реконструкцией с использованием васкуляризированного химерного мышечного гребня
подвздошной кости с участком внутренней косой мышцы (►рис. 3a–d). необходимый
объем, длина и общая морфология свободного лоскута были получены с высокой
точностью в соответствии с 3D-биомедицинской моделью и индивидуально
подобранными хирургическими шаблонами. это оптимизировало реконструктивную
хирургию, поскольку разрезы могли быть меньше, с меньшей кровопотерей и
значительным улучшением общего обзора хирурга. Этот клинический случай
иллюстрирует, как использование 3D-печатных моделей в челюстно-лицевой хирургии
может определить окончательный результат (►рис. 4), сочетая успешную резекцию
инфицированной правой верхней челюсти с очень положительным исходом и
превосходным эстетическим результатом. Реконструкция нижней челюсти При
планировании реконструктивных процедур лица конечной целью реконструкции нижней
челюсти является восстановление речи, жевательной функции, глотания и дыхания, а
также сохранение черт лица. Современные процедуры реконструкции сочетают
фиксацию пластин для реконструкции нижней челюсти и использование
микрососудистых лоскутов.1 При сложных реконструкциях нижней челюсти с
использованием свободных лоскутов для достижения хорошего функционального и
эстетического результата знание точных трехмерных характеристик лоскута является
решающим фактором для восстановления симметрии и структурной целостности .34
таким образом, используя специальное программное обеспечение, можно рассчитать
точные контуры, углы, длину и морфологию новой нижней челюсти.3 Цифровое
отражение здоровой нижней челюсти с аномальной стороны помогает определить
идеальную геометрию скелета. затем, с помощью биомоделей, напечатанных на
3D-принтере, происходит улучшение

позиционирование нижнечелюстных сегментов, сокращение времени операции за счет
отсутствия повторных интраоперационных сгибаний и адаптаций пластин, а также
меньшая вероятность послеоперационной поломки пластины. использование исходной
поверхности кортикальной кости в качестве шаблона для адаптации реконструктивной
пластины, облегчение предоперационного хирургического моделирования и
восстановление центральной окклюзии пациента были некоторыми из преимуществ
виртуального предоперационного планирования. Титановый имплантат, напечатанный
на 3D-принтере, хирургический разрез или остеоэктомия должны точно
соответствовать предоперационному планированию, поскольку имплантаты,
напечатанные на 3D-принтере, настолько прочны, что их нелегко разрезать или
согнуть. поэтому необходимо составить руководство по хирургической остеотомии.10
в некоторых случаях для восстановления лицевого скелета выполняется комбинация
индивидуальных имплантатов и других корректирующих хирургических процедур, таких
как фиксация уцелевших больших кусков сломанной кости, как при разрывных
переломах средней части лица. структура.1 случай 2: левая гемимандибулэктомия с
сохранением мыщелка. Это случай 46-летнего пациента мужского пола с в анамнезе
распространенной плоскоклеточной карциномой полости рта (t3n2bm0), ранее
подвергнутого радикальному иссечению с ипсилатеральной радикальной диссекцией
шеи и адъювантом. лучевая терапия (►рис. 5). предоперационное планирование
методом 3D-биомоделирования (SL) было принято для восстановления симметрии лица
пациента и восстановления/улучшения биофункциональности, например, коррекции
окклюзионного плана и других костных структур. непораженная нижняя челюсть была
зеркально отражена или перевернута и расположена так, чтобы получить наиболее
точный и точный результат. Затем был создан хирургический шаблон для
моделирования костей и пластин (►рис. 6). пациенту была проведена реконструкция
с использованием 3D-биомоделированного костно-мышечного лоскута из гребня
подвздошной кости под контролем каркаса (►рис. 7). Через девять месяцев после
операции, как мы видим на послеоперационных изображениях (►рис. 8), наблюдается
значительное улучшение черт лица пациента за счет восстановления дефекта и
биофункциональности нижней челюсти, что открывает возможность для дентальной
остеоинтегрированной реабилитации на основе имплантатов. . обсуждение Технология
3D-печати — это инновационная медицина, которая еще больше бросает вызов
хирургической практике с точки зрения индивидуального подхода к каждому
пациенту. он может предоставить индивидуальный рис. 1 виртуальное планирование с
использованием программного обеспечения для 3D-реконструкции (anatomics pro,
anatomics TM, Мельбурн, Австралия) на основе данных 3D-КТ. инжир. 2. Клиническое
изображение пациента до операции (а), препарат правой верхней челюсти (б), 3d
биомедицинская модель и хирургические шаблоны для рассечения гребня подвздошной
кости (в).

инжир. 3 субтотальная максилэктомия с резекцией более 50\% неба. (а и б) –
химерные в мышечно-костном лоскуте. (в и г) – имплантация мышечного лоскута
гребня подвздошной кости. инжир. 4 послеоперационных изображения через год: (а и
б) виртуальные 3D-реконструкции; (в и г) вид изнутри и сбоку.

Производство продукта за короткий период времени, что соответствует целям
индивидуализированной медицины, где каждый пациент требует специального,
индивидуально подобранного терапевтического подхода. Конечная цель любой
реконструктивной хирургической процедуры — воспроизвести или улучшить
предоперационную форму и функцию. Применение 3D-печати в
черепно-челюстно-лицевой реконструктивной хирургии меняет подход хирургов к
планированию операций, а дизайнеры разрабатывают индивидуальные
3D-биомедицинские модели. Независимо от того, какой конкретный метод
используется для создания 3D-биомодели, доказаны следующие преимущества
использования 3D-печати в реконструктивной хирургии4,9,10,15,17,35:
1. Путем прямой визуализации анатомических структур и их пространственных
взаимоотношений улучшается понимание сложных основных состояний, что значительно
повышает качество диагностики и планирования лечения.
2. с предварительного согласия пациента пластические хирурги могут лучше
проводить предоперационное консультирование своих пациентов с использованием
3D-моделирования.
3. улучшается предоперационное хирургическое планирование за счет проектирования
разрезов и границ хирургической резекции. 3D-биомодель также позволяет оценить
костные дефекты для трансплантации и адаптации/предварительного изгиба
реконструктивных пластин.
4. Это помогает в разработке инструментов интраоперационного руководства и
улучшает общение между хирургами. это может привести к сокращению времени
операции; сокращение времени проведения общей анестезии; меньшая
продолжительность раневого воздействия; и снижение интраоперационной
кровопотери, ошибок и рисков.
5. он также помогает в производстве индивидуальных имплантатов/протезов в
повседневной хирургической практике, таких как протезы ВНЧС, дистракционные
устройства и фиксирующие устройства; улучшает эстетический результат в
результате индивидуальной подгонки; и дополняет индивидуальные анатомические
потребности. кроме того, индивидуальные имплантаты позволяют избежать
необходимости интраоперационной модификации и настройки, как это происходит со
стандартными имплантатами, что может напрямую привести к улучшению клинических
результатов и снижению риска осложнений, например, инфекций.
6. По сравнению со стандартным имплантатом, изготовленный по индивидуальному
заказу имплантат с большей вероятностью даст превосходные функциональные и
эстетические результаты. типичные материалы для 3D-печати можно стерилизовать с
использованием химикатов, таких как продукты питания и лекарства.
инжир. 5 (а и б) предоперационные снимки больного, вид спереди. (в–д) – данные
предоперационной компьютерной томографии с 3D-реконструкцией. инжир. 6
виртуальное предоперационное планирование с помощью техники 3D-биомоделирования
с определением точного костного дефекта и созданием индивидуального
хирургического шаблона для моделирования кости и пластины
остеосинтеза.

одобренные администрацией протоколы применения глутаральдегида, пар и газ для
интраоперационной обработки. за последнее десятилетие исследователи сообщили о
напечатанных на 3D-принтере протезах носа, ушей, глаз, лица и рук.9 7. Это
образовательный инструмент для студентов-медиков и ординаторов. физические
модели можно реалистично удерживать и вращать, ими можно манипулировать в
интерактивном режиме независимо от сложности, и они доступны без необходимости
использования компьютеров или повышения квалификации.4,36 8. 3D-печать более
предсказуема и обеспечивает точные хирургические результаты. 9. 3D-печать
выгодна не только для восстановления кости, но и для замены мягких тканей,
поскольку можно использовать самые разные материалы.8 рис. 7 (а)
предоперационное планирование свободного лоскута из гребня подвздошной кости;
(b–d) 3D-биомодель анатомии пациента и индивидуально подобранные хирургические
шаблоны; (д) свободный лоскут, подобранный в соответствии с морфологией,
объемом, длиной и углом дефекта пациента; (е) имплантация свободного лоскута из
гребня подвздошной кости. инжир. 8 (а и б) до и через 9 мес после
реконструктивной операции. 3D-печать как инновационная технология также имеет
некоторые ограничения, которые следует учитывать. К основным ограничениям
относятся высокая стоимость и сложность, а также необходимость в
специализированном оборудовании и расходных материалах, таких как фотостойкие
смолы.3 Быстрое прототипирование может применяться только к конструкциям, не
превышающим определенных размеров, поскольку 3D-принтеры не способны производить
чрезвычайно большие , модели всего тела. время, необходимое для создания
3D-модели, также ограничивает ее использование в хирургии плановыми случаями и
делает ее непригодной для неотложных случаев (онкология и острая травма).8,34
еще одним недостатком является более высокая доза радиации, которой приходится
подвергать пациента из-за к конкретным критериям компьютерной томографии, по
сравнению с обычной компьютерной томографией для диагностики.3,34 эти
ограничения могут быть преодолены будущими технологическими разработками.
Заключение: черепные и челюстно-лицевые структуры не только сложны, но и
уникальны среди людей. Использование этих трехмерных биомедицинских моделей в
черепной и челюстно-лицевой реконструктивной хирургии оказалось очень полезным
при проектировании и изготовлении индивидуальных протезов и определении размеров
костных трансплантатов, а также позволило изготовить каркасы для регенерации
кости, а также в других аспектах. медицинского образования и исследований. по
данным Зенха и др.,34 использование этих 3D-биомедицинских моделей особенно
полезно при крупных вторичных дефектах (онкологических, остеорадионекрозе,
травмах) с соответствующими искажениями черепно-лицевой структуры; врожденные
пороки развития и хирургическое вмешательство на первичных опухолях, когда
размеры опухоли значительно изменили нормальную анатомию. Имплантаты,
изготовленные по индивидуальному заказу для реконструкции черепно-лицевых
дефектов, сохраняющие первоначальную анатомию пациентов, приобрели важное
значение из-за лучших характеристик — функциональных и эстетических — по
сравнению с их обычными аналогами.1,3 однако, несмотря на сообщаемые результаты,
дополнительный прогресс в 3d технологии печати необходимы для повышения
разрешения без ущерба для формы, прочности и возможностей рук 3D-биомоделей. в
2014 году национальные институты здравоохранения создали биржу 3D-печати
(3dprint.nih.gov) для содействия обмену файлами 3D-печати с открытым исходным
кодом для медицинских и анатомических моделей и индивидуальных заказов.36,37
интеграция тканей, полученных от пациентов, и индуцированных плюрипотентных
Стволовые клетки с быстрым прогрессом в области 3D-печати биологических тканей и
материалов отлично подходят для биопечати органов с использованием нескольких
печатающих головок, которые будут наносить различные типы клеток
(органоспецифичные, кровеносные сосуды и мышечные клетки), необходимая функция
для изготовления целых гетероклеточных тканей. и органы как потенциальное
решение проблемы нехватки органов для трансплантации.38 за последнее десятилетие
создание трехмерных биомедицинских моделей с использованием изображений было
признано полезным инструментом в различных областях медицины с множеством
применений. недавно в 3D-печати было представлено четвертое измерение: время.
Благодаря этому четвертому измерению времени 4D-печать без особых усилий
позволяет получить сложные пространственно-временные анатомические детали и
может еще больше улучшить предоперационное планирование по сравнению с
3D-печатью.9,39 Благодарности Авторы выражают благодарность всем представителям
пластических, реконструктивных и отделению черепно-челюстно-лицевой хирургии
госпиталя Вила-Нова-де-Гайя/Эспиньо (врачам, медсестрам и секретарям) за
поддержку в лечении этих пациентов.
